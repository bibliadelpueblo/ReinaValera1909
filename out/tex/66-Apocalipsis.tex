\hypertarget{el-tuxedtulo-que-indica-el-origen-el-propuxf3sito-y-el-significado-de-esta-fuente}{%
\subsection{El título que indica el origen, el propósito y el
significado de esta
fuente}\label{el-tuxedtulo-que-indica-el-origen-el-propuxf3sito-y-el-significado-de-esta-fuente}}

\hypertarget{section}{%
\section{1}\label{section}}

\bibverse{1} La revelación de Jesucristo, que Dios le dió, para
manifestar á sus siervos las cosas que deben suceder presto; y la
declaró, enviándola por su ángel á Juan su siervo, \bibverse{2} El cual
ha dado testimonio de la palabra de Dios, y del testimonio de
Jesucristo, y de todas las cosas que ha visto.

\bibverse{3} Bienaventurado el que lee, y los que oyen las palabras de
esta profecía, y guardan las cosas en ella escritas: porque el tiempo
está cerca.

\hypertarget{bendiciones-a-las-siete-parroquias-de-la-provincia-romana-de-asia}{%
\subsection{Bendiciones a las siete parroquias de la provincia romana de
Asia}\label{bendiciones-a-las-siete-parroquias-de-la-provincia-romana-de-asia}}

\bibverse{4} Juan á las siete iglesias que están en Asia: Gracia sea con
vosotros, y paz del que es y que era y que ha de venir, y de los siete
Espíritus que están delante de su trono; \footnote{\textbf{1:4} Éxod
  3,14-15; Apoc 3,1; Apoc 5,6} \bibverse{5} Y de Jesucristo, el testigo
fiel, el primogénito de los muertos, y príncipe de los reyes de la
tierra. Al que nos amó, y nos ha lavado de nuestros pecados con su
sangre, \footnote{\textbf{1:5} Apoc 3,14; Juan 18,37; 1Tim 6,13; Col
  1,18} \bibverse{6} Y nos ha hecho reyes y sacerdotes para Dios y su
Padre; á él sea gloria é imperio para siempre jamás. Amén. \footnote{\textbf{1:6}
  Apoc 5,10; 1Pe 2,5; 1Pe 2,9; Éxod 19,6}

\hypertarget{resumen-anuncio-de-la-venida-de-jesuxfas-al-juicio}{%
\subsection{Resumen: Anuncio de la venida de Jesús al
juicio}\label{resumen-anuncio-de-la-venida-de-jesuxfas-al-juicio}}

\bibverse{7} He aquí que viene con las nubes, y todo ojo le verá, y los
que le traspasaron; y todos los linajes de la tierra se lamentarán sobre
él. Así sea. Amén. \footnote{\textbf{1:7} Mat 24,30; Zac 12,10; Juan
  19,37}

\bibverse{8} Yo soy el Alpha y la Omega, principio y fin, dice el Señor,
que es y que era y que ha de venir, el Todopoderoso. \footnote{\textbf{1:8}
  Is 41,4; Apoc 4,8; Apoc 21,6}

\hypertarget{la-primera-vista-en-patmos-de-las-siete-estrellas-y-los-siete-candeleros-cuxf3mo-estuxe1-preparado-el-seuxf1or-para-venir-a-su-cristianismo-llamar-a-john-para-que-escriba-las-visiones}{%
\subsection{La primera vista (en Patmos) de las siete estrellas y los
siete candeleros: Cómo está preparado el Señor para venir a su
cristianismo; Llamar a John para que escriba las
visiones}\label{la-primera-vista-en-patmos-de-las-siete-estrellas-y-los-siete-candeleros-cuxf3mo-estuxe1-preparado-el-seuxf1or-para-venir-a-su-cristianismo-llamar-a-john-para-que-escriba-las-visiones}}

\bibverse{9} Yo Juan, vuestro hermano, y participante en la tribulación
y en el reino, y en la paciencia de Jesucristo, estaba en la isla que es
llamada Patmos, por la palabra de Dios y el testimonio de Jesucristo.
\bibverse{10} Yo fuí en el Espíritu en el día del Señor, y oí detrás de
mí una gran voz como de trompeta, \bibverse{11} Que decía: Yo soy el
Alpha y Omega, el primero y el último. Escribe en un libro lo que ves, y
envíalo á las siete iglesias que están en Asia; á Efeso, y á Smirna, y á
Pérgamo, y á Tiatira, y á Sardis, y á Filadelfia, y á Laodicea.

\bibverse{12} Y me volví á ver la voz que hablaba conmigo: y vuelto, vi
siete candeleros de oro; \bibverse{13} Y en medio de los siete
candeleros, uno semejante al Hijo del hombre, vestido de una ropa que
llegaba hasta los pies, y ceñido por los pechos con una cinta de oro.
\bibverse{14} Y su cabeza y sus cabellos eran blancos como la lana
blanca, como la nieve; y sus ojos como llama de fuego; \footnote{\textbf{1:14}
  Dan 7,9; Apoc 2,18; Apoc 19,12} \bibverse{15} Y sus pies semejantes al
latón fino, ardientes como en un horno; y su voz como ruido de muchas
aguas. \bibverse{16} Y tenía en su diestra siete estrellas: y de su boca
salía una espada aguda de dos filos. Y su rostro era como el sol cuando
resplandece en su fuerza. \bibverse{17} Y cuando yo le vi, caí como
muerto á sus pies. Y él puso su diestra sobre mí, diciéndome: No temas:
yo soy el primero y el último;

\bibverse{18} Y el que vivo, y he sido muerto; y he aquí que vivo por
siglos de siglos, Amén. Y tengo las llaves del infierno y de la muerte:
\bibverse{19} Escribe las cosas que has visto, y las que son, y las que
han de ser después de éstas: \bibverse{20} El misterio de las siete
estrellas que has visto en mi diestra, y los siete candeleros de oro.
Las siete estrellas son los ángeles de las siete iglesias; y los siete
candeleros que has visto, son las siete iglesias.

\hypertarget{carta-a-la-iglesia-en-uxe9feso-no-dejes-el-primer-amor}{%
\subsection{Carta a la iglesia en Éfeso: ``¡No dejes el primer
amor!''}\label{carta-a-la-iglesia-en-uxe9feso-no-dejes-el-primer-amor}}

\hypertarget{section-1}{%
\section{2}\label{section-1}}

\bibverse{1} Escribe al ángel de la iglesia en EFESO: El que tiene las
siete estrellas en su diestra, el cual anda en medio de los siete
candeleros de oro, dice estas cosas: \footnote{\textbf{2:1} Hech 18,19}

\bibverse{2} Yo sé tus obras, y tu trabajo y paciencia; y que tú no
puedes sufrir los malos, y has probado á los que se dicen ser apóstoles,
y no lo son, y los has hallado mentirosos; \footnote{\textbf{2:2} 1Jn
  4,1} \bibverse{3} Y has sufrido, y has tenido paciencia, y has
trabajado por mi nombre, y no has desfallecido. \bibverse{4} Pero tengo
contra ti que has dejado tu primer amor. \footnote{\textbf{2:4} 1Tim
  5,12} \bibverse{5} Recuerda por tanto de dónde has caído, y
arrepiéntete, y haz las primeras obras; pues si no, vendré presto á ti,
y quitaré tu candelero de su lugar, si no te hubieres arrepentido.
\bibverse{6} Mas tienes esto, que aborreces los hechos de los
Nicolaítas; los cuales yo también aborrezco. \bibverse{7} El que tiene
oído, oiga lo que el Espíritu dice á las iglesias. Al que venciere, daré
á comer del árbol de la vida, el cual está en medio del paraíso de Dios.
\footnote{\textbf{2:7} Apoc 22,2; Gén 2,9}

\hypertarget{carta-a-la-comunidad-de-esmirna-suxe9-fiel-hasta-la-muerte}{%
\subsection{Carta a la comunidad de Esmirna: ``¡Sé fiel hasta la
muerte!''}\label{carta-a-la-comunidad-de-esmirna-suxe9-fiel-hasta-la-muerte}}

\bibverse{8} Y escribe al ángel de la iglesia en SMIRNA: El primero y
postrero, que fué muerto, y vivió, dice estas cosas: \footnote{\textbf{2:8}
  Apoc 1,11; Apoc 1,18}

\bibverse{9} Yo sé tus obras, y tu tribulación, y tu pobreza (pero tú
eres rico), y la blasfemia de los que se dicen ser Judíos, y no lo son,
mas son sinagoga de Satanás. \footnote{\textbf{2:9} Sant 2,5; Apoc 3,9}
\bibverse{10} No tengas ningún temor de las cosas que has de padecer. He
aquí, el diablo ha de enviar algunos de vosotros á la cárcel, para que
seáis probados, y tendréis tribulación de diez días. Sé fiel hasta la
muerte, y yo te daré la corona de la vida. \footnote{\textbf{2:10} Mat
  10,19; Mat 10,28; Apoc 3,11; 2Tim 4,8} \bibverse{11} El que tiene
oído, oiga lo que el Espíritu dice á las iglesias. El que venciere, no
recibirá daño de la muerte segunda. \footnote{\textbf{2:11} Apoc 20,14}

\hypertarget{carta-a-la-parroquia-de-puxe9rgamo-no-sigas-al-mundo-vanidoso}{%
\subsection{Carta a la parroquia de Pérgamo: ``¡No sigas al mundo
vanidoso!''}\label{carta-a-la-parroquia-de-puxe9rgamo-no-sigas-al-mundo-vanidoso}}

\bibverse{12} Y escribe al ángel de la iglesia en PÉRGAMO: El que tiene
la espada aguda de dos filos, dice estas cosas: \footnote{\textbf{2:12}
  Heb 4,12}

\bibverse{13} Yo sé tus obras, y dónde moras, donde está la silla de
Satanás; y retienes mi nombre, y no has negado mi fe, aun en los días en
que fué Antipas mi testigo fiel, el cual ha sido muerto entre vosotros,
donde Satanás mora. \bibverse{14} Pero tengo unas pocas cosas contra ti:
porque tú tienes ahí los que tienen la doctrina de Balaam, el cual
enseñaba á Balac á poner escándalo delante de los hijos de Israel, á
comer de cosas sacrificadas á los ídolos, y á cometer fornicación.
\footnote{\textbf{2:14} Núm 31,16; Jds 1,11; 2Pe 2,15} \bibverse{15} Así
también tú tienes á los que tienen la doctrina de los Nicolaítas, lo
cual yo aborrezco. \bibverse{16} Arrepiéntete, porque de otra manera
vendré á ti presto, y pelearé contra ellos con la espada de mi boca.
\bibverse{17} El que tiene oído, oiga lo que el Espíritu dice á las
iglesias. Al que venciere, daré á comer del maná escondido, y le daré
una piedrecita blanca, y en la piedrecita un nombre nuevo escrito, el
cual ninguno conoce sino aquel que lo recibe.

\hypertarget{carta-a-la-iglesia-de-tiatira-examina-sabiamente-todo-espuxedritu}{%
\subsection{Carta a la iglesia de Tiatira: ``¡Examina sabiamente todo
espíritu!''}\label{carta-a-la-iglesia-de-tiatira-examina-sabiamente-todo-espuxedritu}}

\bibverse{18} Y escribe al ángel de la iglesia en TIATIRA: El Hijo de
Dios, que tiene sus ojos como llama de fuego, y sus pies semejantes al
latón fino, dice estas cosas: \footnote{\textbf{2:18} Hech 16,14; Apoc
  1,14-15}

\bibverse{19} Yo he conocido tus obras, y caridad, y servicio, y fe, y
tu paciencia, y que tus obras postreras son más que las primeras.
\bibverse{20} Mas tengo unas pocas cosas contra ti: porque permites
aquella mujer Jezabel (que se dice profetisa) enseñar, y engañar á mis
siervos, á fornicar, y á comer cosas ofrecidas á los ídolos.
\bibverse{21} Y le he dado tiempo para que se arrepienta de la
fornicación; y no se ha arrepentido. \bibverse{22} He aquí, yo la echo
en cama, y á los que adulteran con ella, en muy grande tribulación, si
no se arrepintieren de sus obras: \bibverse{23} Y mataré á sus hijos con
muerte; y todas las iglesias sabrán que yo soy el que escudriño los
riñones y los corazones: y daré á cada uno de vosotros según sus obras.
\footnote{\textbf{2:23} Sal 7,10; Jer 17,10} \bibverse{24} Pero yo digo
á vosotros, y á los demás que estáis en Tiatira, cualesquiera que no
tienen esta doctrina, y que no han conocido las profundidades de
Satanás, como dicen: Yo no enviaré sobre vosotros otra carga.
\bibverse{25} Empero la que tenéis, tenedla hasta que yo venga.
\bibverse{26} Y al que hubiere vencido, y hubiere guardado mis obras
hasta el fin, yo le daré potestad sobre las gentes; \bibverse{27} Y las
regirá con vara de hierro, y serán quebrantados como vaso de alfarero,
como también yo he recibido de mi Padre: \footnote{\textbf{2:27} Sal
  2,8-9}

\bibverse{28} Y le daré la estrella de la mañana. \bibverse{29} El que
tiene oído, oiga lo que el Espíritu dice á las iglesias.

\hypertarget{carta-a-la-comunidad-de-sardis-no-estuxe9s-muerto-como-los-demuxe1s}{%
\subsection{Carta a la comunidad de Sardis: ``¡No estés muerto como los
demás!''}\label{carta-a-la-comunidad-de-sardis-no-estuxe9s-muerto-como-los-demuxe1s}}

\hypertarget{section-2}{%
\section{3}\label{section-2}}

\bibverse{1} Y escribe al ángel de la iglesia en SARDIS: El que tiene
los siete Espíritus de Dios, y las siete estrellas, dice estas cosas: Yo
conozco tus obras, que tienes nombre que vives, y estás muerto.

\bibverse{2} Sé vigilante y confirma las otras cosas que están para
morir; porque no he hallado tus obras perfectas delante de Dios.
\footnote{\textbf{3:2} Luc 22,32} \bibverse{3} Acuérdate pues de lo que
has recibido y has oído, y guárdalo, y arrepiéntete. Y si no velares,
vendré á ti como ladrón, y no sabrás en qué hora vendré á ti.
\footnote{\textbf{3:3} 1Tes 5,2} \bibverse{4} Mas tienes unas pocas
personas en Sardis que no han ensuciado sus vestiduras: y andarán
conmigo en vestiduras blancas; porque son dignos. \footnote{\textbf{3:4}
  Jds 1,23} \bibverse{5} El que venciere, será vestido de vestiduras
blancas; y no borraré su nombre del libro de la vida, y confesaré su
nombre delante de mi Padre, y delante de sus ángeles. \footnote{\textbf{3:5}
  Apoc 7,13; Mat 10,32; Luc 10,20} \bibverse{6} El que tiene oído, oiga
lo que el Espíritu dice á las iglesias.

\hypertarget{carta-a-la-iglesia-de-filadelfia-afuxe9rrate-a-tu-corona}{%
\subsection{Carta a la Iglesia de Filadelfia: ``¡Aférrate a tu
corona!''}\label{carta-a-la-iglesia-de-filadelfia-afuxe9rrate-a-tu-corona}}

\bibverse{7} Y escribe al ángel de la iglesia en FILADELFIA: Estas cosas
dice el Santo, el Verdadero, el que tiene la llave de David, el que abre
y ninguno cierra, y cierra y ninguno abre: \footnote{\textbf{3:7} Is
  22,22}

\bibverse{8} Yo conozco tus obras: he aquí, he dado una puerta abierta
delante de ti, la cual ninguno puede cerrar; porque tienes un poco de
potencia, y has guardado mi palabra, y no has negado mi nombre.
\bibverse{9} He aquí, yo doy de la sinagoga de Satanás, los que se dicen
ser Judíos, y no lo son, mas mienten; he aquí, yo los constreñiré á que
vengan y adoren delante de tus pies, y sepan que yo te he amado.
\bibverse{10} Porque has guardado la palabra de mi paciencia, yo también
te guardaré de la hora de la tentación que ha de venir en todo el mundo,
para probar á los que moran en la tierra. \footnote{\textbf{3:10} Apoc
  14,12; Mat 6,13} \bibverse{11} He aquí, yo vengo presto; retén lo que
tienes, para que ninguno tome tu corona. \footnote{\textbf{3:11} Apoc
  2,10} \bibverse{12} Al que venciere, yo lo haré columna en el templo
de mi Dios, y nunca más saldrá fuera; y escribiré sobre él el nombre de
mi Dios, y el nombre de la ciudad de mi Dios, la nueva Jerusalem, la
cual desciende del cielo de con mi Dios, y mi nombre nuevo. \footnote{\textbf{3:12}
  Apoc 14,1; Apoc 22,4; Apoc 21,2} \bibverse{13} El que tiene oído, oiga
lo que el Espíritu dice á las iglesias.

\hypertarget{carta-a-la-congregaciuxf3n-de-laodicea-no-seas-tibio-y-lento-para-encontrarte}{%
\subsection{Carta a la congregación de Laodicea: ``¡No seas tibio y
lento para
encontrarte!''}\label{carta-a-la-congregaciuxf3n-de-laodicea-no-seas-tibio-y-lento-para-encontrarte}}

\bibverse{14} Y escribe al ángel de la iglesia en LAODICEA: He aquí dice
el Amén, el testigo fiel y verdadero, el principio de la creación de
Dios:

\bibverse{15} Yo conozco tus obras, que ni eres frío, ni caliente.
¡Ojalá fueses frío, ó caliente! \bibverse{16} Mas porque eres tibio, y
no frío ni caliente, te vomitaré de mi boca. \bibverse{17} Porque tú
dices: Yo soy rico, y estoy enriquecido, y no tengo necesidad de ninguna
cosa; y no conoces que tú eres un cuitado y miserable y pobre y ciego y
desnudo; \footnote{\textbf{3:17} 1Cor 3,18; 1Cor 4,8} \bibverse{18} Yo
te amonesto que de mí compres oro afinado en fuego, para que seas hecho
rico, y seas vestido de vestiduras blancas, para que no se descubra la
vergüenza de tu desnudez; y unge tus ojos con colirio, para que veas.
\footnote{\textbf{3:18} Is 55,1} \bibverse{19} Yo reprendo y castigo á
todos los que amo: sé pues celoso, y arrepiéntete. \footnote{\textbf{3:19}
  Prov 3,12; Heb 12,6; 1Cor 11,32} \bibverse{20} He aquí, yo estoy á la
puerta y llamo: si alguno oyere mi voz y abriere la puerta, entraré á
él, y cenaré con él, y él conmigo. \footnote{\textbf{3:20} Juan 14,23}
\bibverse{21} Al que venciere, yo le daré que se siente conmigo en mi
trono; así como yo he vencido, y me he sentado con mi Padre en su trono.
\footnote{\textbf{3:21} Mat 19,28}

\bibverse{22} El que tiene oído, oiga lo que el Espíritu dice á las
iglesias.

\hypertarget{dios-el-padre-y-el-concilio-celestial-en-el-saluxf3n-del-trono-de-dios}{%
\subsection{Dios el Padre y el concilio celestial en el salón del trono
de
Dios}\label{dios-el-padre-y-el-concilio-celestial-en-el-saluxf3n-del-trono-de-dios}}

\hypertarget{section-3}{%
\section{4}\label{section-3}}

\bibverse{1} Después de estas cosas miré, y he aquí una puerta abierta
en el cielo: y la primera voz que oí, era como de trompeta que hablaba
conmigo, diciendo: Sube acá, y yo te mostraré las cosas que han de ser
después de éstas.

\bibverse{2} Y luego yo fuí en Espíritu: y he aquí, un trono que estaba
puesto en el cielo, y sobre el trono estaba uno sentado. \footnote{\textbf{4:2}
  Is 6,1; Sal 47,9} \bibverse{3} Y el que estaba sentado, era al parecer
semejante á una piedra de jaspe y de sardio: y un arco celeste había
alrededor del trono, semejante en el aspecto á la esmeralda. \footnote{\textbf{4:3}
  Ezeq 1,26-28} \bibverse{4} Y alrededor del trono había veinticuatro
sillas: y vi sobre las sillas veinticuatro ancianos sentados, vestidos
de ropas blancas; y tenían sobre sus cabezas coronas de oro.
\bibverse{5} Y del trono salían relámpagos y truenos y voces: y siete
lámparas de fuego estaban ardiendo delante del trono, las cuales son los
siete Espíritus de Dios. \footnote{\textbf{4:5} Éxod 19,16; Apoc 1,4}
\bibverse{6} Y delante del trono había como un mar de vidrio semejante
al cristal; y en medio del trono, y alrededor del trono, cuatro animales
llenos de ojos delante y detrás. \footnote{\textbf{4:6} Ezeq 1,5; Ezeq
  1,10; Ezeq 1,22; Ezeq 10,14} \bibverse{7} Y el primer animal era
semejante á un león; y el segundo animal, semejante á un becerro; y el
tercer animal tenía la cara como de hombre; y el cuarto animal,
semejante á un águila volando. \bibverse{8} Y los cuatro animales tenían
cada uno por sí seis alas alrededor, y de dentro estaban llenos de ojos;
y no tenían reposo día ni noche, diciendo: Santo, santo, santo el Señor
Dios Todopoderoso, que era, y que es, y que ha de venir. \footnote{\textbf{4:8}
  Is 6,2-3; Éxod 3,14}

\bibverse{9} Y cuando aquellos animales daban gloria y honra y alabanza
al que estaba sentado en el trono, al que vive para siempre jamás,
\bibverse{10} Los veinticuatro ancianos se postraban delante del que
estaba sentado en el trono, y adoraban al que vive para siempre jamás, y
echaban sus coronas delante del trono, diciendo: \bibverse{11} Señor,
digno eres de recibir gloria y honra y virtud: porque tú criaste todas
las cosas, y por tu voluntad tienen ser y fueron criadas.

\hypertarget{el-libro-del-destino-con-los-siete-sellos-y-el-cordero-de-dios-en-el-trono}{%
\subsection{El libro del destino con los siete sellos y el Cordero de
Dios en el
trono}\label{el-libro-del-destino-con-los-siete-sellos-y-el-cordero-de-dios-en-el-trono}}

\hypertarget{section-4}{%
\section{5}\label{section-4}}

\bibverse{1} Y vi en la mano derecha del que estaba sentado sobre el
trono un libro escrito de dentro y de fuera, sellado con siete sellos.
\bibverse{2} Y vi un fuerte ángel predicando en alta voz: ¿Quién es
digno de abrir el libro, y de desatar sus sellos? \bibverse{3} Y ninguno
podía, ni en el cielo, ni en la tierra, ni debajo de la tierra, abrir el
libro, ni mirarlo. \bibverse{4} Y yo lloraba mucho, porque no había sido
hallado ninguno digno de abrir el libro, ni de leerlo, ni de mirarlo.
\bibverse{5} Y uno de los ancianos me dice: No llores: he aquí el león
de la tribu de Judá, la raíz de David, que ha vencido para abrir el
libro, y desatar sus siete sellos. \footnote{\textbf{5:5} Gén 49,9-10;
  Is 11,1}

\bibverse{6} Y miré; y he aquí en medio del trono y de los cuatro
animales, y en medio de los ancianos, estaba un Cordero como inmolado,
que tenía siete cuernos, y siete ojos, que son los siete Espíritus de
Dios enviados en toda la tierra. \footnote{\textbf{5:6} Is 53,7; Juan
  1,29} \bibverse{7} Y él vino, y tomó el libro de la mano derecha de
aquel que estaba sentado en el trono. \bibverse{8} Y cuando hubo tomado
el libro, los cuatro animales y los veinticuatro ancianos se postraron
delante del Cordero, teniendo cada uno arpas, y copas de oro llenas de
perfumes, que son las oraciones de los santos: \bibverse{9} Y cantaban
un nuevo cántico, diciendo: Digno eres de tomar el libro, y de abrir sus
sellos; porque tú fuiste inmolado, y nos has redimido para Dios con tu
sangre, de todo linaje y lengua y pueblo y nación; \footnote{\textbf{5:9}
  Sal 98,1} \bibverse{10} Y nos has hecho para nuestro Dios reyes y
sacerdotes, y reinaremos sobre la tierra. \footnote{\textbf{5:10} Apoc
  1,6; Éxod 19,6}

\hypertarget{el-himno-de-alabanza-de-los-uxe1ngeles-y-de-toda-la-creaciuxf3n}{%
\subsection{El himno de alabanza de los ángeles y de toda la
creación}\label{el-himno-de-alabanza-de-los-uxe1ngeles-y-de-toda-la-creaciuxf3n}}

\bibverse{11} Y miré, y oí voz de muchos ángeles alrededor del trono, y
de los animales, y de los ancianos; y la multitud de ellos era millones
de millones, \footnote{\textbf{5:11} Heb 12,22} \bibverse{12} Que decían
en alta voz: El Cordero que fué inmolado es digno de tomar el poder y
riquezas y sabiduría, y fortaleza y honra y gloria y alabanza.
\footnote{\textbf{5:12} 1Cró 29,11; Fil 2,9-10}

\bibverse{13} Y oí á toda criatura que está en el cielo, y sobre la
tierra, y debajo de la tierra, y que está en el mar, y todas las cosas
que en ellos están, diciendo: Al que está sentado en el trono, y al
Cordero, sea la bendición, y la honra, y la gloria, y el poder, para
siempre jamás.

\bibverse{14} Y los cuatro animales decían: Amén. Y los veinticuatro
ancianos cayeron sobre sus rostros, y adoraron al que vive para siempre
jamás.

\hypertarget{la-apertura-de-los-primeros-seis-sellos-por-el-cordero-los-cuatro-jinetes}{%
\subsection{La apertura de los primeros seis sellos por el Cordero; Los
cuatro
jinetes}\label{la-apertura-de-los-primeros-seis-sellos-por-el-cordero-los-cuatro-jinetes}}

\hypertarget{section-5}{%
\section{6}\label{section-5}}

\bibverse{1} Y miré cuando el Cordero abrió uno de los sellos, y oí á
uno de los cuatro animales diciendo como con una voz de trueno: Ven y
ve. \bibverse{2} Y miré, y he aquí un caballo blanco: y el que estaba
sentado encima de él, tenía un arco; y le fué dada una corona, y salió
victorioso, para que también venciese. \footnote{\textbf{6:2} Zac 6,1-5}

\bibverse{3} Y cuando él abrió el segundo sello, oí al segundo animal,
que decía: Ven y ve. \bibverse{4} Y salió otro caballo bermejo: y al que
estaba sentado sobre él, fué dado poder de quitar la paz de la tierra, y
que se maten unos á otros: y fuéle dada una grande espada.

\bibverse{5} Y cuando él abrió el tercer sello, oí al tercer animal, que
decía: Ven y ve. Y miré, y he aquí un caballo negro: y el que estaba
sentado encima de él, tenía un peso en su mano. \bibverse{6} Y oí una
voz en medio de los cuatro animales, que decía: Dos libras de trigo por
un denario, y seis libras de cebada por un denario: y no hagas daño al
vino ni al aceite.

\bibverse{7} Y cuando él abrió el cuarto sello, oí la voz del cuarto
animal, que decía: Ven y ve. \bibverse{8} Y miré, y he aquí un caballo
amarillo: y el que estaba sentado sobre él tenía por nombre Muerte; y el
infierno le seguía: y le fué dada potestad sobre la cuarta parte de la
tierra, para matar con espada, con hambre, con mortandad, y con las
bestias de la tierra. \footnote{\textbf{6:8} Ezeq 14,21}

\hypertarget{los-muxe1rtires}{%
\subsection{Los mártires}\label{los-muxe1rtires}}

\bibverse{9} Y cuando él abrió el quinto sello, vi debajo del altar las
almas de los que habían sido muertos por la palabra de Dios y por el
testimonio que ellos tenían. \bibverse{10} Y clamaban en alta voz
diciendo: ¿Hasta cuándo, Señor, santo y verdadero, no juzgas y vengas
nuestra sangre de los que moran en la tierra? \bibverse{11} Y les fueron
dadas sendas ropas blancas, y fuéles dicho que reposasen todavía un poco
de tiempo, hasta que se completaran sus consiervos y sus hermanos, que
también habían de ser muertos como ellos.

\hypertarget{platos-terribles-en-la-imagen-de-los-fenuxf3menos-naturales}{%
\subsection{Platos terribles en la imagen de los fenómenos
naturales}\label{platos-terribles-en-la-imagen-de-los-fenuxf3menos-naturales}}

\bibverse{12} Y miré cuando él abrió el sexto sello, y he aquí fué hecho
un gran terremoto; y el sol se puso negro como un saco de cilicio, y la
luna se puso toda como sangre; \bibverse{13} Y las estrellas del cielo
cayeron sobre la tierra, como la higuera echa sus higos cuando es movida
de gran viento. \footnote{\textbf{6:13} Is 34,4} \bibverse{14} Y el
cielo se apartó como un libro que es envuelto; y todo monte y las islas
fueron movidas de sus lugares. \bibverse{15} Y los reyes de la tierra, y
los príncipes, y los ricos, y los capitanes, y los fuertes, y todo
siervo y todo libre, se escondieron en las cuevas y entre las peñas de
los montes; \bibverse{16} Y decían á los montes y á las peñas: Caed
sobre nosotros, y escondednos de la cara de aquél que está sentado sobre
el trono, y de la ira del Cordero: \footnote{\textbf{6:16} Luc 23,30}
\bibverse{17} Porque el gran día de su ira es venido; ¿y quién podrá
estar firme? \footnote{\textbf{6:17} Am 5,18; Rom 2,5; Mal 3,2}

\hypertarget{el-sellamiento-de-una-selecciuxf3n-de-144.000-de-las-doce-tribus-de-israel}{%
\subsection{El sellamiento de una selección (de 144.000 de las doce
tribus de
Israel)}\label{el-sellamiento-de-una-selecciuxf3n-de-144.000-de-las-doce-tribus-de-israel}}

\hypertarget{section-6}{%
\section{7}\label{section-6}}

\bibverse{1} Y después de estas cosas vi cuatro ángeles que estaban
sobre los cuatro ángulos de la tierra, deteniendo los cuatro vientos de
la tierra, para que no soplase viento sobre la tierra, ni sobre la mar,
ni sobre ningún árbol. \footnote{\textbf{7:1} Dan 7,2} \bibverse{2} Y vi
otro ángel que subía del nacimiento del sol, teniendo el sello del Dios
vivo: y clamó con gran voz á los cuatro ángeles, á los cuales era dado
hacer daño á la tierra y á la mar, \bibverse{3} Diciendo: No hagáis daño
á la tierra, ni al mar, ni á los árboles, hasta que señalemos á los
siervos de nuestro Dios en sus frentes. \bibverse{4} Y oí el número de
los señalados: ciento cuarenta y cuatro mil señalados de todas las
tribus de los hijos de Israel. \footnote{\textbf{7:4} Apoc 14,1; Apoc
  14,3} \bibverse{5} De la tribu de Judá, doce mil señalados. De la
tribu de Rubén, doce mil señalados. De la tribu de Gad, doce mil
señalados. \bibverse{6} De la tribu de Aser, doce mil señalados. De la
tribu de Neftalí, doce mil señalados. De la tribu de Manasés, doce mil
señalados. \bibverse{7} De la tribu de Simeón, doce mil señalados. De la
tribu de Leví, doce mil señalados. De la tribu de Issachâr, doce mil
señalados. \bibverse{8} De la tribu de Zabulón, doce mil señalados. De
la tribu de José, doce mil señalados. De la tribu de Benjamín, doce mil
señalados.

\hypertarget{el-homenaje-a-las-innumerables-probadas-y-probadas-almas-creyentes-de-todos-los-pueblos-ante-el-trono-de-dios-que-han-salido-de-la-gran-tribulaciuxf3n}{%
\subsection{El homenaje a las innumerables, probadas y probadas almas
creyentes de todos los pueblos ante el trono de Dios, que han salido de
la gran
tribulación}\label{el-homenaje-a-las-innumerables-probadas-y-probadas-almas-creyentes-de-todos-los-pueblos-ante-el-trono-de-dios-que-han-salido-de-la-gran-tribulaciuxf3n}}

\bibverse{9} Después de estas cosas miré, y he aquí una gran compañía,
la cual ninguno podía contar, de todas gentes y linajes y pueblos y
lenguas, que estaban delante del trono y en la presencia del Cordero,
vestidos de ropas blancas, y palmas en sus manos; \bibverse{10} Y
clamaban en alta voz, diciendo: Salvación á nuestro Dios que está
sentado sobre el trono, y al Cordero.

\bibverse{11} Y todos los ángeles estaban alrededor del trono, y de los
ancianos y los cuatro animales; y postráronse sobre sus rostros delante
del trono, y adoraron á Dios, \bibverse{12} Diciendo: Amén: La bendición
y la gloria y la sabiduría, y la acción de gracias y la honra y la
potencia y la fortaleza, sean á nuestro Dios para siempre jamás. Amén.

\bibverse{13} Y respondió uno de los ancianos, diciéndome: Estos que
están vestidos de ropas blancas, ¿quiénes son, y de dónde han venido?

\bibverse{14} Y yo le dije: Señor, tú lo sabes. Y él me dijo: Estos son
los que han venido de grande tribulación, y han lavado sus ropas, y las
han blanqueado en la sangre del Cordero.

\bibverse{15} Por esto están delante del trono de Dios, y le sirven día
y noche en su templo: y el que está sentado en el trono tenderá su
pabellón sobre ellos. \bibverse{16} No tendrán más hambre, ni sed, y el
sol no caerá más sobre ellos, ni otro ningún calor. \footnote{\textbf{7:16}
  Is 49,10} \bibverse{17} Porque el Cordero que está en medio del trono
los pastoreará, y los guiará á fuentes vivas de aguas: y Dios limpiará
toda lágrima de los ojos de ellos. \footnote{\textbf{7:17} Sal 23,2;
  Apoc 21,4; Is 25,8}

\hypertarget{la-soluciuxf3n-del-suxe9ptimo-sello-y-el-silencio-en-el-cielo-presentando-los-juicios-de-las-trompetas}{%
\subsection{La solución del séptimo sello y el silencio en el cielo
(presentando los juicios de las
trompetas)}\label{la-soluciuxf3n-del-suxe9ptimo-sello-y-el-silencio-en-el-cielo-presentando-los-juicios-de-las-trompetas}}

\hypertarget{section-7}{%
\section{8}\label{section-7}}

\bibverse{1} Y cuando él abrió el séptimo sello, fué hecho silencio en
el cielo casi por media hora. \footnote{\textbf{8:1} Zac 2,17; Hab 2,20}
\bibverse{2} Y vi los siete ángeles que estaban delante de Dios; y les
fueron dadas siete trompetas. \footnote{\textbf{8:2} Mat 24,31}

\bibverse{3} Y otro ángel vino, y se paró delante del altar, teniendo un
incensario de oro; y le fué dado mucho incienso para que lo añadiese á
las oraciones de todos los santos sobre el altar de oro que estaba
delante del trono. \bibverse{4} Y el humo del incienso subió de la mano
del ángel delante de Dios, con las oraciones de los santos. \footnote{\textbf{8:4}
  Sal 141,2} \bibverse{5} Y el ángel tomó el incensario, y lo llenó del
fuego del altar, y echólo en la tierra; y fueron hechos truenos y voces
y relámpagos y terremotos. \footnote{\textbf{8:5} Ezeq 10,2}

\bibverse{6} Y los siete ángeles que tenían las siete trompetas, se
aparejaron para tocar.

\hypertarget{las-primeras-cuatro-trompetas-es-decir-las-plagas-que-vienen-de-arriba}{%
\subsection{Las primeras cuatro trompetas (es decir, las plagas que
vienen de
arriba)}\label{las-primeras-cuatro-trompetas-es-decir-las-plagas-que-vienen-de-arriba}}

\bibverse{7} Y el primer ángel tocó la trompeta, y fué hecho granizo y
fuego, mezclado con sangre, y fueron arrojados á la tierra; y la tercera
parte de los árboles fué quemada, y quemóse toda la hierba verde.
\footnote{\textbf{8:7} Éxod 9,23-26}

\bibverse{8} Y el segundo ángel tocó la trompeta, y como un grande monte
ardiendo con fuego fué lanzado en la mar; y la tercera parte de la mar
se tornó en sangre. \footnote{\textbf{8:8} Éxod 7,20-21} \bibverse{9} Y
murió la tercera parte de las criaturas que estaban en la mar, las
cuales tenían vida; y la tercera parte de los navíos pereció.

\bibverse{10} Y el tercer ángel tocó la trompeta, y cayó del cielo una
grande estrella, ardiendo como una antorcha, y cayó en la tercera parte
de los ríos, y en las fuentes de las aguas. \footnote{\textbf{8:10} Is
  14,12} \bibverse{11} Y el nombre de la estrella se dice Ajenjo. Y la
tercera parte de las aguas fué vuelta en ajenjo: y muchos hombres
murieron por las aguas, porque fueron hechas amargas.

\bibverse{12} Y el cuarto ángel tocó la trompeta, y fué herida la
tercera parte del sol, y la tercera parte de la luna, y la tercera parte
de las estrellas; de tal manera que se oscureció la tercera parte de
ellos, y no alumbraba la tercera parte del día, y lo mismo de la noche.

\hypertarget{la-primera-llamada-del-uxe1guila-y-la-quinta-y-sexta-trompetas-las-dos-plagas-que-vienen-de-abajo-es-decir-del-infierno}{%
\subsection{La primera llamada del águila y la quinta y sexta trompetas
(las dos plagas que vienen de abajo, es decir, del
infierno)}\label{la-primera-llamada-del-uxe1guila-y-la-quinta-y-sexta-trompetas-las-dos-plagas-que-vienen-de-abajo-es-decir-del-infierno}}

\bibverse{13} Y miré, y oí un ángel volar por medio del cielo, diciendo
en alta voz: ¡Ay! ¡ay! ¡ay! de los que moran en la tierra, por razón de
las otras voces de trompeta de los tres ángeles que han de tocar!

\hypertarget{la-quinta-trompeta-o-el-primer-ay}{%
\subsection{La quinta trompeta o el primer
ay}\label{la-quinta-trompeta-o-el-primer-ay}}

\hypertarget{section-8}{%
\section{9}\label{section-8}}

\bibverse{1} Y el quinto ángel tocó la trompeta, y vi una estrella que
cayó del cielo en la tierra; y le fué dada la llave del pozo del abismo.
\bibverse{2} Y abrió el pozo del abismo, y subió humo del pozo como el
humo de un gran horno; y oscurecióse el sol y el aire por el humo del
pozo. \footnote{\textbf{9:2} Jl 2,2; Jl 2,10} \bibverse{3} Y del humo
salieron langostas sobre la tierra; y fuéles dada potestad, como tienen
potestad los escorpiones de la tierra. \bibverse{4} Y les fué mandado
que no hiciesen daño á la hierba de la tierra, ni á ninguna cosa verde,
ni á ningún árbol, sino solamente á los hombres que no tienen la señal
de Dios en sus frentes. \bibverse{5} Y les fué dado que no los matasen,
sino que los atormentasen cinco meses; y su tormento era como tormento
de escorpión, cuando hiere al hombre. \bibverse{6} Y en aquellos días
buscarán los hombres la muerte, y no la hallarán; y desearán morir, y la
muerte huirá de ellos. \footnote{\textbf{9:6} Apoc 6,16}

\bibverse{7} Y el parecer de las langostas era semejante á caballos
aparejados para la guerra: y sobre sus cabezas tenían como coronas
semejantes al oro; y sus caras como caras de hombres. \bibverse{8} Y
tenían cabellos como cabellos de mujeres: y sus dientes eran como
dientes de leones. \bibverse{9} Y tenían corazas como corazas de hierro;
y el estruendo de sus alas, como el ruido de carros que con muchos
caballos corren á la batalla. \bibverse{10} Y tenían colas semejantes á
las de los escorpiones, y tenían en sus colas aguijones; y su poder era
de hacer daño á los hombres cinco meses. \bibverse{11} Y tienen sobre sí
por rey al ángel del abismo, cuyo nombre en hebraico es Abaddon, y en
griego, Apollyon.

\bibverse{12} El primer ¡Ay! es pasado: he aquí, vienen aún dos ayes
después de estas cosas.

\hypertarget{la-sexta-trompeta-o-el-segundo-ay}{%
\subsection{La sexta trompeta o el segundo
ay}\label{la-sexta-trompeta-o-el-segundo-ay}}

\bibverse{13} Y el sexto ángel tocó la trompeta; y oí una voz de los
cuatro cuernos del altar de oro que estaba delante de Dios,
\bibverse{14} Diciendo al sexto ángel que tenía la trompeta: Desata los
cuatro ángeles que están atados en el gran río Eufrates. \footnote{\textbf{9:14}
  Apoc 16,12}

\bibverse{15} Y fueron desatados los cuatro ángeles que estaban
aparejados para la hora y día y mes y año, para matar la tercera parte
de los hombres. \footnote{\textbf{9:15} Apoc 8,-1} \bibverse{16} Y el
número del ejército de los de á caballo era doscientos millones. Y oí el
número de ellos. \bibverse{17} Y así vi los caballos en visión, y los
que sobre ellos estaban sentados, los cuales tenían corazas de fuego, de
jacinto, y de azufre. Y las cabezas de los caballos eran como cabezas de
leones; y de la boca de ellos salía fuego y humo y azufre. \bibverse{18}
De estas tres plagas fué muerta la tercera parte de los hombres: del
fuego, y del humo, y del azufre, que salían de la boca de ellos.
\bibverse{19} Porque su poder está en su boca y en sus colas: porque sus
colas eran semejantes á serpientes, y tenían cabezas, y con ellas dañan.

\bibverse{20} Y los otros hombres que no fueron muertos con estas
plagas, aun no se arrepintieron de las obras de sus manos, para que no
adorasen á los demonios, y á las imágenes de oro, y de plata, y de
metal, y de piedra, y de madera; las cuales no pueden ver, ni oir, ni
andar: \footnote{\textbf{9:20} Apoc 16,9}

\bibverse{21} Y no se arrepintieron de sus homicidios, ni de sus
hechicerías, ni de su fornicación, ni de sus hurtos.

\hypertarget{un-uxe1ngel-sostiene-un-libro-abierto-y-jura}{%
\subsection{Un ángel sostiene un libro abierto y
jura}\label{un-uxe1ngel-sostiene-un-libro-abierto-y-jura}}

\hypertarget{section-9}{%
\section{10}\label{section-9}}

\bibverse{1} Y vi otro ángel fuerte descender del cielo, cercado de una
nube, y el arco celeste sobre su cabeza; y su rostro era como el sol, y
sus pies como columnas de fuego. \bibverse{2} Y tenía en su mano un
librito abierto: y puso su pie derecho sobre el mar, y el izquierdo
sobre la tierra; \bibverse{3} Y clamó con grande voz, como cuando un
león ruge: y cuando hubo clamado, siete truenos hablaron sus voces.
\bibverse{4} Y cuando los siete truenos hubieron hablado sus voces, yo
iba á escribir, y oí una voz del cielo que me decía: Sella las cosas que
los siete truenos han hablado, y no las escribas. \footnote{\textbf{10:4}
  Dan 12,4; Dan 12,9; Sal 29,-1}

\bibverse{5} Y el ángel que vi estar sobre el mar y sobre la tierra,
levantó su mano al cielo, \bibverse{6} Y juró por el que vive para
siempre jamás, que ha criado el cielo y las cosas que están en él, y la
tierra y las cosas que están en ella, y el mar y las cosas que están en
él, que el tiempo no será más. \bibverse{7} Pero en los días de la voz
del séptimo ángel, cuando él comenzare á tocar la trompeta, el misterio
de Dios será consumado, como él lo anunció á sus siervos los profetas.
\footnote{\textbf{10:7} Apoc 11,15; Hech 3,21}

\hypertarget{johannes-consume-el-librito-agridulce}{%
\subsection{Johannes consume el librito
agridulce}\label{johannes-consume-el-librito-agridulce}}

\bibverse{8} Y la voz que oí del cielo hablaba otra vez conmigo, y
decía: Ve, y toma el librito abierto de la mano del ángel que está sobre
el mar y sobre la tierra.

\bibverse{9} Y fuí al ángel, diciéndole que me diese el librito, y él me
dijo: Toma, y trágalo; y él te hará amargar tu vientre, pero en tu boca
será dulce como la miel.

\bibverse{10} Y tomé el librito de la mano del ángel, y lo devoré; y era
dulce en mi boca como la miel; y cuando lo hube devorado, fué amargo mi
vientre. \bibverse{11} Y él me dice: Necesario es que otra vez
profetices á muchos pueblos y gentes y lenguas y reyes. \footnote{\textbf{10:11}
  Jer 1,10}

\hypertarget{la-medida-del-templo-la-preservaciuxf3n-de-los-fieles-durante-la-intensa-visita-de-los-gentiles-a-la-ciudad-santa}{%
\subsection{La medida del templo; la preservación de los fieles durante
la intensa visita de los gentiles a la ciudad
santa}\label{la-medida-del-templo-la-preservaciuxf3n-de-los-fieles-durante-la-intensa-visita-de-los-gentiles-a-la-ciudad-santa}}

\hypertarget{section-10}{%
\section{11}\label{section-10}}

\bibverse{1} Y me fué dada una caña semejante á una vara, y se me dijo:
Levántate, y mide el templo de Dios, y el altar, y á los que adoran en
él. \footnote{\textbf{11:1} Ezeq 40,3; Ezeq 42,20; Zac 2,5-6}
\bibverse{2} Y echa fuera el patio que está fuera del templo, y no lo
midas, porque es dado á los Gentiles; y hollarán la ciudad santa
cuarenta y dos meses. \footnote{\textbf{11:2} Luc 21,24}

\hypertarget{efectividad-muerte-y-ascensiuxf3n-de-los-dos-testigos-de-dios}{%
\subsection{Efectividad, muerte y ascensión de los dos testigos de
Dios}\label{efectividad-muerte-y-ascensiuxf3n-de-los-dos-testigos-de-dios}}

\bibverse{3} Y daré á mis dos testigos, y ellos profetizarán por mil
doscientos y sesenta días, vestidos de sacos. \footnote{\textbf{11:3}
  Apoc 12,6}

\bibverse{4} Estas son las dos olivas, y los dos candeleros que están
delante del Dios de la tierra. \footnote{\textbf{11:4} Zac 4,3; Zac
  4,11-14} \bibverse{5} Y si alguno les quisiere dañar, sale fuego de la
boca de ellos, y devora á sus enemigos: y si alguno les quisiere hacer
daño, es necesario que él sea así muerto. \bibverse{6} Estos tienen
potestad de cerrar el cielo, que no llueva en los días de su profecía, y
tienen poder sobre las aguas para convertirlas en sangre, y para herir
la tierra con toda plaga cuantas veces quisieren.

\bibverse{7} Y cuando ellos hubieren acabado su testimonio, la bestia
que sube del abismo hará guerra contra ellos, y los vencerá, y los
matará. \footnote{\textbf{11:7} Apoc 13,1; Apoc 13,7} \bibverse{8} Y sus
cuerpos serán echados en las plazas de la grande ciudad, que
espiritualmente es llamada Sodoma y Egipto, donde también nuestro Señor
fué crucificado. \bibverse{9} Y los de los linajes, y de los pueblos, y
de las lenguas, y de los Gentiles verán los cuerpos de ellos por tres
días y medio, y no permitirán que sus cuerpos sean puestos en sepulcros.
\bibverse{10} Y los moradores de la tierra se gozarán sobre ellos, y se
alegrarán, y se enviarán dones los unos á los otros; porque estos dos
profetas han atormentado á los que moran sobre la tierra.

\bibverse{11} Y después de tres días y medio el espíritu de vida enviado
de Dios, entró en ellos, y se alzaron sobre sus pies, y vino gran temor
sobre los que los vieron. \bibverse{12} Y oyeron una grande voz del
cielo, que les decía: Subid acá. Y subieron al cielo en una nube, y sus
enemigos los vieron. \bibverse{13} Y en aquella hora fué hecho gran
temblor de tierra, y la décima parte de la ciudad cayó, y fueron muertos
en el temblor de tierra en número de siete mil hombres: y los demás
fueron espantados, y dieron gloria al Dios del cielo.

\bibverse{14} El segundo ¡Ay! es pasado: he aquí, el tercer ¡Ay! vendrá
presto.

\hypertarget{la-suxe9ptima-trompeta-juxfabilo-de-victoria-en-el-cielo-la-apariciuxf3n-del-arca}{%
\subsection{La séptima trompeta; Júbilo de victoria en el cielo; la
aparición del
arca}\label{la-suxe9ptima-trompeta-juxfabilo-de-victoria-en-el-cielo-la-apariciuxf3n-del-arca}}

\bibverse{15} Y el séptimo ángel tocó la trompeta, y fueron hechas
grandes voces en el cielo, que decían: Los reinos del mundo han venido á
ser los reinos de nuestro Señor, y de su Cristo: y reinará para siempre
jamás.

\bibverse{16} Y los veinticuatro ancianos que estaban sentados delante
de Dios en sus sillas, se postraron sobre sus rostros, y adoraron á
Dios, \footnote{\textbf{11:16} Apoc 4,4; Apoc 4,10} \bibverse{17}
Diciendo: Te damos gracias, Señor Dios Todopoderoso, que eres y que eras
y que has de venir, porque has tomado tu grande potencia, y has reinado.
\bibverse{18} Y se han airado las naciones, y tu ira es venida, y el
tiempo de los muertos, para que sean juzgados, y para que des el
galardón á tus siervos los profetas, y á los santos, y á los que temen
tu nombre, á los pequeñitos y á los grandes, y para que destruyas los
que destruyen la tierra.

\bibverse{19} Y el templo de Dios fué abierto en el cielo, y el arca de
su testamento fué vista en su templo. Y fueron hechos relámpagos y voces
y truenos y terremotos y grande granizo. \footnote{\textbf{11:19} Apoc
  15,5}

\hypertarget{la-mujer-sol-y-el-draguxf3n-rescate-de-la-mujer-y-su-hijo-reciuxe9n-nacido}{%
\subsection{La Mujer Sol y el Dragón; Rescate de la mujer y su hijo
recién
nacido}\label{la-mujer-sol-y-el-draguxf3n-rescate-de-la-mujer-y-su-hijo-reciuxe9n-nacido}}

\hypertarget{section-11}{%
\section{12}\label{section-11}}

\bibverse{1} Y una grande señal apareció en el cielo: una mujer vestida
del sol, y la luna debajo de sus pies, y sobre su cabeza una corona de
doce estrellas. \bibverse{2} Y estando preñada, clamaba con dolores de
parto, y sufría tormento por parir.

\bibverse{3} Y fué vista otra señal en el cielo: y he aquí un grande
dragón bermejo, que tenía siete cabezas y diez cuernos, y en sus cabezas
siete diademas. \bibverse{4} Y su cola arrastraba la tercera parte de
las estrellas del cielo, y las echó en tierra. Y el dragón se paró
delante de la mujer que estaba para parir, á fin de devorar á su hijo
cuando hubiese parido. \bibverse{5} Y ella parió un hijo varón, el cual
había de regir todas las gentes con vara de hierro: y su hijo fué
arrebatado para Dios y á su trono. \footnote{\textbf{12:5} Sal 2,9}
\bibverse{6} Y la mujer huyó al desierto, donde tiene lugar aparejado de
Dios, para que allí la mantengan mil doscientos y sesenta días.
\footnote{\textbf{12:6} Apoc 19,2; Gén 3,1; Luc 10,18}

\hypertarget{la-victoria-de-michael-sobre-el-draguxf3n-en-el-cielo-cauxedda-del-draguxf3n-himno-celestial-de-alabanza-el-reinado-de-dios-y-su-ungido-estuxe1-amaneciendo}{%
\subsection{La victoria de Michael sobre el dragón en el cielo; Caída
del dragón; himno celestial de alabanza; el reinado de Dios y su ungido
está
amaneciendo}\label{la-victoria-de-michael-sobre-el-draguxf3n-en-el-cielo-cauxedda-del-draguxf3n-himno-celestial-de-alabanza-el-reinado-de-dios-y-su-ungido-estuxe1-amaneciendo}}

\bibverse{7} Y fué hecha una grande batalla en el cielo: Miguel y sus
ángeles lidiaban contra el dragón; y lidiaba el dragón y sus ángeles,
\bibverse{8} Y no prevalecieron, ni su lugar fué más hallado en el
cielo. \bibverse{9} Y fué lanzado fuera aquel gran dragón, la serpiente
antigua, que se llama Diablo y Satanás, el cual engaña á todo el mundo;
fué arrojado en tierra, y sus ángeles fueron arrojados con él.

\bibverse{10} Y oí una grande voz en el cielo que decía: Ahora ha venido
la salvación, y la virtud, y el reino de nuestro Dios, y el poder de su
Cristo; porque el acusador de nuestros hermanos ha sido arrojado, el
cual los acusaba delante de nuestro Dios día y noche. \footnote{\textbf{12:10}
  Apoc 11,15} \bibverse{11} Y ellos le han vencido por la sangre del
Cordero, y por la palabra de su testimonio; y no han amado sus vidas
hasta la muerte. \footnote{\textbf{12:11} Apoc 6,9; Apoc 7,14}
\bibverse{12} Por lo cual alegraos, cielos, y los que moráis en ellos.
¡Ay de los moradores de la tierra y del mar! porque el diablo ha
descendido á vosotros, teniendo grande ira, sabiendo que tiene poco
tiempo.

\hypertarget{persecuciuxf3n-infructuosa-de-mujeres-por-parte-del-draguxf3n}{%
\subsection{Persecución infructuosa de mujeres por parte del
dragón}\label{persecuciuxf3n-infructuosa-de-mujeres-por-parte-del-draguxf3n}}

\bibverse{13} Y cuando vió el dragón que él había sido arrojado á la
tierra, persiguió á la mujer que había parido al hijo varón.
\bibverse{14} Y fueron dadas á la mujer dos alas de grande águila, para
que de la presencia de la serpiente volase al desierto, á su lugar,
donde es mantenida por un tiempo, y tiempos, y la mitad de un tiempo.
\bibverse{15} Y la serpiente echó de su boca tras la mujer agua como un
río, á fin de hacer que fuese arrebatada del río. \bibverse{16} Y la
tierra ayudó á la mujer, y la tierra abrió su boca, y sorbió el río que
había echado el dragón de su boca. \bibverse{17} Entonces el dragón fué
airado contra la mujer; y se fué á hacer guerra contra los otros de la
simiente de ella, los cuales guardan los mandamientos de Dios, y tienen
el testimonio de Jesucristo.

\hypertarget{la-primera-bestia-del-mar-de-diez-cuernos-y-siete-cabezas-amonestaciuxf3n-a-perseverar}{%
\subsection{La primera bestia del mar, de diez cuernos y siete cabezas;
Amonestación a
perseverar}\label{la-primera-bestia-del-mar-de-diez-cuernos-y-siete-cabezas-amonestaciuxf3n-a-perseverar}}

\hypertarget{section-12}{%
\section{13}\label{section-12}}

\bibverse{1} Y yo me paré sobre la arena del mar, y vi una bestia subir
del mar, que tenía siete cabezas y diez cuernos; y sobre sus cuernos
diez diademas; y sobre las cabezas de ella nombre de blasfemia.
\footnote{\textbf{13:1} Dan 7,3-7} \bibverse{2} Y la bestia que vi, era
semejante á un leopardo, y sus pies como de oso, y su boca como boca de
león. Y el dragón le dió su poder, y su trono, y grande potestad.
\bibverse{3} Y vi una de sus cabezas como herida de muerte, y la llaga
de su muerte fué curada: y se maravilló toda la tierra en pos de la
bestia. \bibverse{4} Y adoraron al dragón que había dado la potestad á
la bestia, y adoraron á la bestia, diciendo: ¿Quién es semejante á la
bestia, y quién podrá lidiar con ella?

\bibverse{5} Y le fué dada boca que hablaba grandes cosas y blasfemias:
y le fué dada potencia de obrar cuarenta y dos meses. \bibverse{6} Y
abrió su boca en blasfemias contra Dios, para blasfemar su nombre, y su
tabernáculo, y á los que moran en el cielo. \bibverse{7} Y le fué dado
hacer guerra contra los santos, y vencerlos. También le fué dada
potencia sobre toda tribu y pueblo y lengua y gente. \footnote{\textbf{13:7}
  Apoc 11,7; Dan 7,21} \bibverse{8} Y todos los que moran en la tierra
le adoraron, cuyos nombres no están escritos en el libro de la vida del
Cordero, el cual fué muerto desde el principio del mundo. \bibverse{9}
Si alguno tiene oído, oiga. \bibverse{10} El que lleva en cautividad, va
en cautividad: el que á cuchillo matare, es necesario que á cuchillo sea
muerto. Aquí está la paciencia y la fe de los santos.

\bibverse{11} Después vi otra bestia que subía de la tierra; y tenía dos
cuernos semejantes á los de un cordero, mas hablaba como un dragón.
\bibverse{12} Y ejerce todo el poder de la primera bestia en presencia
de ella; y hace á la tierra y á los moradores de ella adorar la primera
bestia, cuya llaga de muerte fué curada. \bibverse{13} Y hace grandes
señales, de tal manera que aun hace descender fuego del cielo á la
tierra delante de los hombres. \footnote{\textbf{13:13} Mat 24,24; 2Tes
  2,9} \bibverse{14} Y engaña á los moradores de la tierra por las
señales que le ha sido dado hacer en presencia de la bestia, mandando á
los moradores de la tierra que hagan la imagen de la bestia que tiene la
herida de cuchillo, y vivió. \bibverse{15} Y le fué dado que diese
espíritu á la imagen de la bestia, para que la imagen de la bestia
hable; y hará que cualesquiera que no adoraren la imagen de la bestia
sean muertos. \bibverse{16} Y hacía que á todos, á los pequeños y
grandes, ricos y pobres, libres y siervos, se pusiese una marca en su
mano derecha, ó en sus frentes: \bibverse{17} Y que ninguno pudiese
comprar ó vender, sino el que tuviera la señal, ó el nombre de la
bestia, ó el número de su nombre. \bibverse{18} Aquí hay sabiduría. El
que tiene entendimiento, cuente el número de la bestia; porque es el
número de hombre: y el número de ella, seiscientos sesenta y seis.
\footnote{\textbf{13:18} Apoc 15,2}

\hypertarget{el-cordero-y-la-iglesia-perfecta-de-los-144.000-en-el-monte-sion}{%
\subsection{El Cordero y la iglesia perfecta de los 144.000 en el monte
Sion}\label{el-cordero-y-la-iglesia-perfecta-de-los-144.000-en-el-monte-sion}}

\hypertarget{section-13}{%
\section{14}\label{section-13}}

\bibverse{1} Y miré, y he aquí, el Cordero estaba sobre el monte de
Sión, y con él ciento cuarenta y cuatro mil, que tenían el nombre de su
Padre escrito en sus frentes. \footnote{\textbf{14:1} Apoc 7,4; Apoc
  3,12} \bibverse{2} Y oí una voz del cielo como ruido de muchas aguas,
y como sonido de un gran trueno: y oí una voz de tañedores de arpas que
tañían con sus arpas: \footnote{\textbf{14:2} Apoc 1,15} \bibverse{3} Y
cantaban como un cántico nuevo delante del trono, y delante de los
cuatro animales, y de los ancianos: y ninguno podía aprender el cántico
sino aquellos ciento cuarenta y cuatro mil, los cuales fueron comprados
de entre los de la tierra. \bibverse{4} Estos son los que con mujeres no
fueron contaminados; porque son vírgenes. Estos, los que siguen al
Cordero por donde quiera que fuere. Estos fueron comprados de entre los
hombres por primicias para Dios y para el Cordero. \bibverse{5} Y en sus
bocas no ha sido hallado engaño; porque ellos son sin mácula delante del
trono de Dios.

\hypertarget{tres-llamados-de-uxe1ngeles-proclaman-un-mensaje-eterno-de-salvaciuxf3n-para-todos-los-pueblos-asuxed-como-la-cauxedda-de-babilonia-y-anuncian-el-juicio-de-los-adoradores-de-la-bestia}{%
\subsection{Tres llamados de ángeles proclaman un mensaje eterno de
salvación para todos los pueblos, así como la caída de Babilonia y
anuncian el juicio de los adoradores de la
bestia}\label{tres-llamados-de-uxe1ngeles-proclaman-un-mensaje-eterno-de-salvaciuxf3n-para-todos-los-pueblos-asuxed-como-la-cauxedda-de-babilonia-y-anuncian-el-juicio-de-los-adoradores-de-la-bestia}}

\bibverse{6} Y vi otro ángel volar por en medio del cielo, que tenía el
evangelio eterno para predicarlo á los que moran en la tierra, y á toda
nación y tribu y lengua y pueblo, \bibverse{7} Diciendo en alta voz:
Temed á Dios, y dadle honra; porque la hora de su juicio es venida; y
adorad á aquel que ha hecho el cielo y la tierra y el mar y las fuentes
de las aguas.

\bibverse{8} Y otro ángel le siguió, diciendo: Ha caído, ha caído
Babilonia, aquella grande ciudad, porque ella ha dado á beber á todas
las naciones del vino del furor de su fornicación. \footnote{\textbf{14:8}
  Apoc 18,-1; Jer 25,15-16; Jer 51,7; Is 21,9}

\bibverse{9} Y el tercer ángel los siguió, diciendo en alta voz: Si
alguno adora á la bestia y á su imagen, y toma la señal en su frente, ó
en su mano, \footnote{\textbf{14:9} Apoc 13,12-17} \bibverse{10} Este
también beberá del vino de la ira de Dios, el cual está echado puro en
el cáliz de su ira; y será atormentado con fuego y azufre delante de los
santos ángeles, y delante del Cordero: \footnote{\textbf{14:10} Sal 75,9}
\bibverse{11} Y el humo del tormento de ellos sube para siempre jamás. Y
los que adoran á la bestia y á su imagen, no tienen reposo día ni noche,
ni cualquiera que tomare la señal de su nombre.

\bibverse{12} Aquí está la paciencia de los santos; aquí están los que
guardan los mandamientos de Dios, y la fe de Jesús.

\hypertarget{una-voz-celestial-proclama-la-bienaventuranza-de-los-creyentes-muxe1rtires-que-son-fieles-hasta-la-muerte}{%
\subsection{Una voz celestial proclama la bienaventuranza de los
creyentes (mártires) que son fieles hasta la
muerte}\label{una-voz-celestial-proclama-la-bienaventuranza-de-los-creyentes-muxe1rtires-que-son-fieles-hasta-la-muerte}}

\bibverse{13} Y oí una voz del cielo que me decía: Escribe:
Bienaventurados los muertos que de aquí adelante mueren en el Señor. Sí,
dice el Espíritu, que descansarán de sus trabajos; porque sus obras con
ellos siguen. \footnote{\textbf{14:13} Is 57,2; Heb 4,10; Fil 1,23}

\hypertarget{el-juicio-del-hijo-del-hombre-en-la-tierra-bajo-la-imagen-de-una-cosecha-de-grano-y-una-vendimia}{%
\subsection{El juicio del Hijo del Hombre en la tierra bajo la imagen de
una cosecha de grano y una
vendimia}\label{el-juicio-del-hijo-del-hombre-en-la-tierra-bajo-la-imagen-de-una-cosecha-de-grano-y-una-vendimia}}

\bibverse{14} Y miré, y he aquí una nube blanca; y sobre la nube uno
sentado semejante al Hijo del hombre, que tenía en su cabeza una corona
de oro, y en su mano una hoz aguda. \footnote{\textbf{14:14} Mar 13,26}
\bibverse{15} Y otro ángel salió del templo, clamando en alta voz al que
estaba sentado sobre la nube: Mete tu hoz, y siega; porque la hora de
segar te es venida, porque la mies de la tierra está madura. \footnote{\textbf{14:15}
  Mat 13,39; Jl 4,13} \bibverse{16} Y el que estaba sentado sobre la
nube echó su hoz sobre la tierra, y la tierra fué segada.

\bibverse{17} Y salió otro ángel del templo que está en el cielo,
teniendo también una hoz aguda. \bibverse{18} Y otro ángel salió del
altar, el cual tenía poder sobre el fuego, y clamó con gran voz al que
tenía la hoz aguda, diciendo: Mete tu hoz aguda, y vendimia los racimos
de la tierra; porque están maduras sus uvas. \bibverse{19} Y el ángel
echó su hoz aguda en la tierra, y vendimió la viña de la tierra, y echó
la uva en el grande lagar de la ira de Dios. \bibverse{20} Y el lagar
fué hollado fuera de la ciudad, y del lagar salió sangre hasta los
frenos de los caballos por mil y seiscientos estadios.

\hypertarget{los-siete-uxe1ngeles-con-las-siete-uxfaltimas-plagas-la-alabanza-de-los-vencedores-en-el-mar-de-cristal}{%
\subsection{Los siete ángeles con las siete últimas plagas; la alabanza
de los vencedores en el mar de
cristal}\label{los-siete-uxe1ngeles-con-las-siete-uxfaltimas-plagas-la-alabanza-de-los-vencedores-en-el-mar-de-cristal}}

\hypertarget{section-14}{%
\section{15}\label{section-14}}

\bibverse{1} Y vi otra señal en el cielo, grande y admirable, que era
siete ángeles que tenían las siete plagas postreras; porque en ellas es
consumada la ira de Dios. \footnote{\textbf{15:1} Apoc 16,1}

\bibverse{2} Y vi así como un mar de vidrio mezclado con fuego; y los
que habían alcanzado la victoria de la bestia, y de su imagen, y de su
señal, y del número de su nombre, estar sobre el mar de vidrio, teniendo
las arpas de Dios. \footnote{\textbf{15:2} Apoc 4,6} \bibverse{3} Y
cantan el cántico de Moisés siervo de Dios, y el cántico del Cordero,
diciendo: Grandes y maravillosas son tus obras, Señor Dios Todopoderoso;
justos y verdaderos son tus caminos, Rey de los santos. \footnote{\textbf{15:3}
  Éxod 15,1; Éxod 15,11; Deut 32,4; Sal 145,17; Jer 10,6-7} \bibverse{4}
¿Quién no te temerá, oh Señor, y engrandecerá tu nombre? porque tú sólo
eres santo; por lo cual todas las naciones vendrán, y adorarán delante
de ti, porque tus juicios son manifestados. \footnote{\textbf{15:4} Sal
  86,9; Jer 16,19-21}

\hypertarget{la-apariciuxf3n-y-el-equipamiento-de-los-siete-uxe1ngeles-de-la-copa-de-la-ira}{%
\subsection{La aparición y el equipamiento de los siete ángeles de la
copa de la
ira}\label{la-apariciuxf3n-y-el-equipamiento-de-los-siete-uxe1ngeles-de-la-copa-de-la-ira}}

\bibverse{5} Y después de estas cosas miré, y he aquí el templo del
tabernáculo del testimonio fué abierto en el cielo; \footnote{\textbf{15:5}
  Apoc 11,19} \bibverse{6} Y salieron del templo siete ángeles, que
tenían siete plagas, vestidos de un lino limpio y blanco, y ceñidos
alrededor de los pechos con bandas de oro.

\bibverse{7} Y uno de los cuatro animales dió á los siete ángeles siete
copas de oro, llenas de la ira de Dios, que vive para siempre jamás.
\bibverse{8} Y fué el templo lleno de humo por la majestad de Dios, y
por su potencia; y ninguno podía entrar en el templo, hasta que fuesen
consumadas las siete plagas de los siete ángeles. \footnote{\textbf{15:8}
  Éxod 40,34; 1Re 8,10; Is 6,4; Ezeq 44,4}

\hypertarget{el-derramamiento-de-los-siete-tazones-de-ira}{%
\subsection{El derramamiento de los siete tazones de
ira}\label{el-derramamiento-de-los-siete-tazones-de-ira}}

\hypertarget{section-15}{%
\section{16}\label{section-15}}

\bibverse{1} Y oí una gran voz del templo, que decía á los siete
ángeles: Id, y derramad las siete copas de la ira de Dios sobre la
tierra.

\bibverse{2} Y fué el primero, y derramó su copa sobre la tierra; y vino
una plaga mala y dañosa sobre los hombres que tenían la señal de la
bestia, y sobre los que adoraban su imagen.

\bibverse{3} Y el segundo ángel derramó su copa sobre el mar, y se
convirtió en sangre como de un muerto; y toda alma viviente fué muerta
en el mar.

\bibverse{4} Y el tercer ángel derramó su copa sobre los ríos, y sobre
las fuentes de las aguas, y se convirtieron en sangre. \footnote{\textbf{16:4}
  Éxod 7,17-21} \bibverse{5} Y oí al ángel de las aguas, que decía:
Justo eres tú, oh Señor, que eres y que eras, el Santo, porque has
juzgado estas cosas: \bibverse{6} Porque ellos derramaron la sangre de
los santos y de los profetas, también tú les has dado á beber sangre;
pues lo merecen.

\bibverse{7} Y oí á otro del altar, que decía: Ciertamente, Señor Dios
Todopoderoso, tus juicios son verdaderos y justos.

\bibverse{8} Y el cuarto ángel derramó su copa sobre el sol; y le fué
dado quemar á los hombres con fuego. \bibverse{9} Y los hombres se
quemaron con el grande calor, y blasfemaron el nombre de Dios, que tiene
potestad sobre estas plagas, y no se arrepintieron para darle gloria.

\bibverse{10} Y el quinto ángel derramó su copa sobre la silla de la
bestia; y su reino se hizo tenebroso, y se mordían sus lenguas de dolor;
\bibverse{11} Y blasfemaron del Dios del cielo por sus dolores, y por
sus plagas, y no se arrepintieron de sus obras.

\bibverse{12} Y el sexto ángel derramó su copa sobre el gran río
Eufrates; y el agua de él se secó, para que fuese preparado el camino de
los reyes del Oriente. \footnote{\textbf{16:12} Is 11,15-16}
\bibverse{13} Y vi salir de la boca del dragón, y de la boca de la
bestia, y de la boca del falso profeta, tres espíritus inmundos á manera
de ranas: \footnote{\textbf{16:13} Apoc 12,3; Éxod 8,3} \bibverse{14}
Porque son espíritus de demonios, que hacen señales, para ir á los reyes
de la tierra y de todo el mundo, para congregarlos para la batalla de
aquel gran día del Dios Todopoderoso.

\bibverse{15} He aquí, yo vengo como ladrón. Bienaventurado el que vela,
y guarda sus vestiduras, para que no ande desnudo, y vean su vergüenza.
\footnote{\textbf{16:15} 1Tes 5,2} \bibverse{16} Y los congregó en el
lugar que en hebreo se llama Armagedón.

\bibverse{17} Y el séptimo ángel derramó su copa por el aire; y salió
una grande voz del templo del cielo, del trono, diciendo: Hecho es.
\bibverse{18} Entonces fueron hechos relámpagos y voces y truenos; y
hubo un gran temblor de tierra, un terremoto tan grande, cual no fué
jamás desde que los hombres han estado sobre la tierra. \bibverse{19} Y
la ciudad grande fué partida en tres partes, y las ciudades de las
naciones cayeron; y la grande Babilonia vino en memoria delante de Dios,
para darle el cáliz del vino del furor de su ira. \bibverse{20} Y toda
isla huyó, y los montes no fueron hallados. \footnote{\textbf{16:20}
  Apoc 6,14} \bibverse{21} Y cayó del cielo sobre los hombres un grande
granizo como del peso de un talento: y los hombres blasfemaron de Dios
por la plaga del granizo; porque su plaga fué muy grande. \footnote{\textbf{16:21}
  Éxod 9,23}

\hypertarget{descripciuxf3n-de-la-espluxe9ndida-pero-abominable-mujer-entronizada-sobre-la-bestia}{%
\subsection{Descripción de la espléndida pero abominable mujer
entronizada sobre la
bestia}\label{descripciuxf3n-de-la-espluxe9ndida-pero-abominable-mujer-entronizada-sobre-la-bestia}}

\hypertarget{section-16}{%
\section{17}\label{section-16}}

\bibverse{1} Y vino uno de los siete ángeles que tenían las siete copas,
y habló conmigo, diciéndome: Ven acá, y te mostraré la condenación de la
grande ramera, la cual está sentada sobre muchas aguas: \footnote{\textbf{17:1}
  Apoc 15,1} \bibverse{2} Con la cual han fornicado los reyes de la
tierra, y los que moran en la tierra se han embriagado con el vino de su
fornicación. \bibverse{3} Y me llevó en Espíritu al desierto; y vi una
mujer sentada sobre una bestia bermeja llena de nombres de blasfemia y
que tenía siete cabezas y diez cuernos. \bibverse{4} Y la mujer estaba
vestida de púrpura y de escarlata, y dorada con oro, y adornada de
piedras preciosas y de perlas, teniendo un cáliz de oro en su mano lleno
de abominaciones, y de la suciedad de su fornicación; \bibverse{5} Y en
su frente un nombre escrito: MISTERIO, BABILONIA LA GRANDE, LA MADRE DE
LAS FORNICACIONES Y DE LAS ABOMINACIONES DE LA TIERRA. \bibverse{6} Y vi
la mujer embriagada de la sangre de los santos, y de la sangre de los
mártires de Jesús: y cuando la vi, quedé maravillado de grande
admiración. \footnote{\textbf{17:6} Apoc 18,24}

\hypertarget{descripciuxf3n-del-animal-de-siete-cabezas-y-diez-cuernos-asuxed-como-su-destino-pasado-y-futuro}{%
\subsection{Descripción del animal de siete cabezas y diez cuernos, así
como su destino pasado y
futuro}\label{descripciuxf3n-del-animal-de-siete-cabezas-y-diez-cuernos-asuxed-como-su-destino-pasado-y-futuro}}

\bibverse{7} Y el ángel me dijo: ¿Por qué te maravillas? Yo te diré el
misterio de la mujer, y de la bestia que la trae, la cual tiene siete
cabezas y diez cuernos. \bibverse{8} La bestia que has visto, fué, y no
es; y ha de subir del abismo, y ha de ir á perdición: y los moradores de
la tierra, cuyos nombres no están escritos en el libro de la vida desde
la fundación del mundo, se maravillarán viendo la bestia que era y no
es, aunque es.

\bibverse{9} Y aquí hay mente que tiene sabiduría. Las siete cabezas son
siete montes, sobre los cuales se asienta la mujer. \bibverse{10} Y son
siete reyes. Los cinco son caídos; el uno es, el otro aun no es venido;
y cuando viniere, es necesario que dure breve tiempo. \bibverse{11} Y la
bestia que era, y no es, es también el octavo, y es de los siete, y va á
perdición. \bibverse{12} Y los diez cuernos que has visto, son diez
reyes, que aun no han recibido reino; mas tomarán potencia por una hora
como reyes con la bestia. \footnote{\textbf{17:12} Apoc 13,1}
\bibverse{13} Estos tienen un consejo, y darán su potencia y autoridad á
la bestia. \bibverse{14} Ellos pelearán contra el Cordero, y el Cordero
los vencerá, porque es el Señor de los señores, y el Rey de los reyes: y
los que están con él son llamados, y elegidos, y fieles.

\hypertarget{interpretaciuxf3n-de-la-imagen}{%
\subsection{Interpretación de la
imagen}\label{interpretaciuxf3n-de-la-imagen}}

\bibverse{15} Y él me dice: Las aguas que has visto donde la ramera se
sienta, son pueblos y muchedumbres y naciones y lenguas. \footnote{\textbf{17:15}
  Is 8,7; Jer 47,2}

\bibverse{16} Y los diez cuernos que viste en la bestia, éstos
aborrecerán á la ramera, y la harán desolada y desnuda: y comerán sus
carnes, y la quemarán con fuego: \bibverse{17} Porque Dios ha puesto en
sus corazones ejecutar lo que le plugo, y el ponerse de acuerdo, y dar
su reino á la bestia, hasta que sean cumplidas las palabras de Dios.
\bibverse{18} Y la mujer que has visto, es la grande ciudad que tiene
reino sobre los reyes de la tierra.

\hypertarget{anuncio-del-juicio-que-cae-sobre-babilonia-la-primera-llamada-del-uxe1ngel}{%
\subsection{Anuncio del juicio que cae sobre Babilonia; La primera
llamada del
ángel}\label{anuncio-del-juicio-que-cae-sobre-babilonia-la-primera-llamada-del-uxe1ngel}}

\hypertarget{section-17}{%
\section{18}\label{section-17}}

\bibverse{1} Y después de estas cosas vi otro ángel descender del cielo
teniendo grande potencia; y la tierra fué alumbrada de su gloria.
\footnote{\textbf{18:1} Ezeq 43,2} \bibverse{2} Y clamó con fortaleza en
alta voz, diciendo: Caída es, caída es la grande Babilonia, y es hecha
habitación de demonios, y guarida de todo espíritu inmundo, y albergue
de todas aves sucias y aborrecibles. \footnote{\textbf{18:2} Apoc 14,8;
  Is 34,11; Is 34,13; Jer 50,39} \bibverse{3} Porque todas las gentes
han bebido del vino del furor de su fornicación; y los reyes de la
tierra han fornicado con ella, y los mercaderes de la tierra se han
enriquecido de la potencia de sus deleites. \footnote{\textbf{18:3} Jer
  51,7; Nah 3,4}

\hypertarget{una-segunda-voz}{%
\subsection{Una segunda voz}\label{una-segunda-voz}}

\bibverse{4} Y oí otra voz del cielo, que decía: Salid de ella, pueblo
mío, porque no seáis participantes de sus pecados, y que no recibáis de
sus plagas; \footnote{\textbf{18:4} Is 48,20; Jer 50,8; Jer 51,6; Jer
  51,45; 2Cor 6,17} \bibverse{5} Porque sus pecados han llegado hasta el
cielo, y Dios se ha acordado de sus maldades. \footnote{\textbf{18:5}
  Gén 18,20-21; Jer 51,9} \bibverse{6} Tornadle á dar como ella os ha
dado, y pagadle al doble según sus obras; en el cáliz que ella os dió á
beber, dadle á beber doblado. \footnote{\textbf{18:6} Jer 50,15; Jer
  50,29; Sal 137,8; 2Tes 1,6} \bibverse{7} Cuanto ella se ha
glorificado, y ha estado en deleites, tanto dadle de tormento y llanto;
porque dice en su corazón: Yo estoy sentada reina, y no soy viuda, y no
veré llanto. \footnote{\textbf{18:7} Is 47,7-9} \bibverse{8} Por lo cual
en un día vendrán sus plagas, muerte, llanto y hambre, y será quemada
con fuego; porque el Señor Dios es fuerte, que la juzgará.

\hypertarget{las-lamentaciones-de-los-reyes-de-la-tierra-de-los-comerciantes-y-marineros-por-la-cauxedda-de-la-ciudad}{%
\subsection{Las lamentaciones de los reyes de la tierra, de los
comerciantes y marineros por la caída de la
ciudad}\label{las-lamentaciones-de-los-reyes-de-la-tierra-de-los-comerciantes-y-marineros-por-la-cauxedda-de-la-ciudad}}

\bibverse{9} Y llorarán y se lamentarán sobre ella los reyes de la
tierra, los cuales han fornicado con ella y han vivido en deleites,
cuando ellos vieren el humo de su incendio, \bibverse{10} Estando lejos
por el temor de su tormento, diciendo: ¡Ay, ay, de aquella gran ciudad
de Babilonia, aquella fuerte ciudad; porque en una hora vino tu juicio!
\bibverse{11} Y los mercaderes de la tierra lloran y se lamentan sobre
ella, porque ninguno compra más sus mercaderías: \footnote{\textbf{18:11}
  Ezeq 27,36} \bibverse{12} Mercadería de oro, y de plata, y de piedras
preciosas, y de margaritas, y de lino fino, y de escarlata, y de seda, y
de grana, y de toda madera olorosa, y de todo vaso de marfil, y de todo
vaso de madera preciosa, y de cobre, y de hierro, y de mármol;
\footnote{\textbf{18:12} Ezeq 27,12-13; Ezeq 27,22} \bibverse{13} Y
canela, y olores, y ungüentos, y de incienso, y de vino, y de aceite; y
flor de harina y trigo, y de bestias, y de ovejas; y de caballos, y de
carros, y de siervos, y de almas de hombres. \bibverse{14} Y los frutos
del deseo de tu alma se apartaron de ti; y todas las cosas gruesas y
excelentes te han faltado, y nunca más las hallarás. \bibverse{15} Los
mercaderes de estas cosas, que se han enriquecido, se pondrán lejos de
ella por el temor de su tormento, llorando y lamentando, \bibverse{16} Y
diciendo: ¡Ay, ay, aquella gran ciudad, que estaba vestida de lino fino,
y de escarlata, y de grana, y estaba dorada con oro, y adornada de
piedras preciosas y de perlas! \footnote{\textbf{18:16} Apoc 17,4}
\bibverse{17} Porque en una hora han sido desoladas tantas riquezas. Y
todo patrón, y todos los que viajan en naves, y marineros, y todos los
que trabajan en el mar, se estuvieron lejos; \footnote{\textbf{18:17}
  Ezeq 27,27-36} \bibverse{18} Y viendo el humo de su incendio, dieron
voces, diciendo: ¿Qué ciudad era semejante á esta gran ciudad?
\bibverse{19} Y echaron polvo sobre sus cabezas; y dieron voces,
llorando y lamentando, diciendo: ¡Ay, ay, de aquella gran ciudad, en la
cual todos los que tenían navíos en la mar se habían enriquecido de sus
riquezas; que en una hora ha sido desolada!

\hypertarget{aclamaciuxf3n-gozosa-la-voz-celestial-a-los-habitantes-celestiales}{%
\subsection{Aclamación gozosa, la voz celestial, a los habitantes
celestiales}\label{aclamaciuxf3n-gozosa-la-voz-celestial-a-los-habitantes-celestiales}}

\bibverse{20} Alégrate sobre ella, cielo, y vosotros, santos, apóstoles,
y profetas; porque Dios ha vengado vuestra causa en ella. \footnote{\textbf{18:20}
  Jer 51,38}

\hypertarget{el-signo-simbuxf3lico-de-la-aniquilaciuxf3n-la-desolaciuxf3n-que-prevalece-en-la-ciudad-destruida}{%
\subsection{El signo simbólico de la aniquilación; la desolación que
prevalece en la ciudad
destruida}\label{el-signo-simbuxf3lico-de-la-aniquilaciuxf3n-la-desolaciuxf3n-que-prevalece-en-la-ciudad-destruida}}

\bibverse{21} Y un ángel fuerte tomó una piedra como una grande piedra
de molino, y la echó en la mar, diciendo: Con tanto ímpetu será
derribada Babilonia, aquella grande ciudad, y nunca jamás será hallada.
\footnote{\textbf{18:21} Jer 51,63-64} \bibverse{22} Y voz de tañedores
de arpas, y de músicos, y de tañedores de flautas y de trompetas, no
será más oída en ti; y todo artífice de cualquier oficio, no será más
hallado en ti; y el sonido de muela no será más en ti oído: \footnote{\textbf{18:22}
  Is 24,8; Ezeq 26,13} \bibverse{23} Y luz de antorcha no alumbrará más
en ti; y voz de esposo ni de esposa no será más en ti oída; porque tus
mercaderes eran los magnates de la tierra; porque en tus hechicerías
todas las gentes han errado. \footnote{\textbf{18:23} Jer 25,10; Is 23,8}
\bibverse{24} Y en ella fué hallada la sangre de los profetas y de los
santos, y de todos los que han sido muertos en la tierra. \footnote{\textbf{18:24}
  Apoc 6,10; Apoc 17,6}

\hypertarget{el-juxfabilo-en-el-cielo-por-la-cauxedda-de-babilonia-la-pruxf3xima-boda-del-cordero}{%
\subsection{El júbilo en el cielo por la caída de Babilonia; la próxima
boda del
Cordero}\label{el-juxfabilo-en-el-cielo-por-la-cauxedda-de-babilonia-la-pruxf3xima-boda-del-cordero}}

\hypertarget{section-18}{%
\section{19}\label{section-18}}

\bibverse{1} Después de estas cosas oí una gran voz de gran compañía en
el cielo, que decía: Aleluya: Salvación y honra y gloria y potencia al
Señor Dios nuestro. \bibverse{2} Porque sus juicios son verdaderos y
justos; porque él ha juzgado á la grande ramera, que ha corrompido la
tierra con su fornicación, y ha vengado la sangre de sus siervos de la
mano de ella.

\bibverse{3} Y otra vez dijeron: Aleluya. Y su humo subió para siempre
jamás. \footnote{\textbf{19:3} Is 34,10} \bibverse{4} Y los veinticuatro
ancianos y los cuatro animales se postraron en tierra, y adoraron á Dios
que estaba sentado sobre el trono, diciendo: Amén: Aleluya. \footnote{\textbf{19:4}
  Apoc 4,4; Apoc 4,6; Apoc 5,11; Sal 106,48}

\bibverse{5} Y salió una voz del trono, que decía: Load á nuestro Dios
todos sus siervos, y los que le teméis, así pequeños como grandes.

\bibverse{6} Y oí como la voz de una grande compañía, y como el ruido de
muchas aguas, y como la voz de grandes truenos, que decía: Aleluya:
porque reinó el Señor nuestro Dios Todopoderoso. \bibverse{7} Gocémonos
y alegrémonos y démosle gloria; porque son venidas las bodas del
Cordero, y su esposa se ha aparejado. \footnote{\textbf{19:7} Apoc 21,9}
\bibverse{8} Y le fué dado que se vista de lino fino, limpio y
brillante: porque el lino fino son las justificaciones de los santos.
\footnote{\textbf{19:8} Is 61,10}

\bibverse{9} Y él me dice: Escribe: Bienaventurados los que son llamados
á la cena del Cordero. Y me dijo: Estas palabras de Dios son verdaderas.

\bibverse{10} Y yo me eché á sus pies para adorarle. Y él me dijo: Mira
que no lo hagas: yo soy siervo contigo, y con tus hermanos que tienen el
testimonio de Jesús: adora á Dios; porque el testimonio de Jesús es el
espíritu de la profecía. \footnote{\textbf{19:10} Apoc 22,8-9}

\hypertarget{la-batalla-del-mesuxedas-la-destrucciuxf3n-de-la-bestia-y-sus-seguidores-es-decir-todos-los-ejuxe9rcitos-o-pueblos-hostiles}{%
\subsection{La batalla del Mesías; la destrucción de la bestia y sus
seguidores, es decir, todos los ejércitos o pueblos
hostiles}\label{la-batalla-del-mesuxedas-la-destrucciuxf3n-de-la-bestia-y-sus-seguidores-es-decir-todos-los-ejuxe9rcitos-o-pueblos-hostiles}}

\bibverse{11} Y vi el cielo abierto; y he aquí un caballo blanco, y el
que estaba sentado sobre él, era llamado Fiel y Verdadero, el cual con
justicia juzga y pelea. \footnote{\textbf{19:11} Apoc 3,14; Mat 24,30;
  Is 11,4-5} \bibverse{12} Y sus ojos eran como llama de fuego, y había
en su cabeza muchas diademas; y tenía un nombre escrito que ninguno
entendía sino él mismo. \footnote{\textbf{19:12} Apoc 1,14; Apoc 3,12}
\bibverse{13} Y estaba vestido de una ropa teñida en sangre: y su nombre
es llamado EL VERBO DE DIOS. \footnote{\textbf{19:13} Is 63,1-2; Juan
  1,1} \bibverse{14} Y los ejércitos que están en el cielo le seguían en
caballos blancos, vestidos de lino finísimo, blanco y limpio.
\footnote{\textbf{19:14} Apoc 17,14} \bibverse{15} Y de su boca sale una
espada aguda, para herir con ella las gentes: y él los regirá con vara
de hierro; y él pisa el lagar del vino del furor, y de la ira del Dios
Todopoderoso. \footnote{\textbf{19:15} Sal 2,9; Apoc 14,19-20}
\bibverse{16} Y en su vestidura y en su muslo tiene escrito este nombre:
REY DE REYES Y SEÑOR DE SEÑORES. \footnote{\textbf{19:16} Apoc 1,5; 1Tim
  6,15}

\bibverse{17} Y vi un ángel que estaba en el sol, y clamó con gran voz,
diciendo á todas las aves que volaban por medio del cielo: Venid, y
congregaos á la cena del gran Dios, \footnote{\textbf{19:17} Ezeq 39,4;
  Ezeq 39,17-20} \bibverse{18} Para que comáis carnes de reyes, y de
capitanes, y carnes de fuertes, y carnes de caballos, y de los que están
sentados sobre ellos; y carnes de todos, libres y siervos, de pequeños y
de grandes. \bibverse{19} Y vi la bestia, y los reyes de la tierra y sus
ejércitos, congregados para hacer guerra contra el que estaba sentado
sobre el caballo, y contra su ejército. \footnote{\textbf{19:19} Apoc
  16,14; Apoc 16,16} \bibverse{20} Y la bestia fué presa, y con ella el
falso profeta que había hecho las señales delante de ella, con las
cuales había engañado á los que tomaron la señal de la bestia, y habían
adorado su imagen. Estos dos fueron lanzados vivos dentro de un lago de
fuego ardiendo en azufre. \footnote{\textbf{19:20} 2Tes 2,8; Apoc
  13,11-17}

\bibverse{21} Y los otros fueron muertos con la espada que salía de la
boca del que estaba sentado sobre el caballo, y todas las aves fueron
hartas de las carnes de ellos.

\hypertarget{encadenamiento-de-satanuxe1s-la-primera-resurrecciuxf3n-y-el-reino-milenario-de-paz}{%
\subsection{Encadenamiento de Satanás; la primera resurrección y el
reino milenario de
paz}\label{encadenamiento-de-satanuxe1s-la-primera-resurrecciuxf3n-y-el-reino-milenario-de-paz}}

\hypertarget{section-19}{%
\section{20}\label{section-19}}

\bibverse{1} Y vi un ángel descender del cielo, que tenía la llave del
abismo, y una grande cadena en su mano. \footnote{\textbf{20:1} Apoc 9,1}
\bibverse{2} Y prendió al dragón, aquella serpiente antigua, que es el
Diablo y Satanás, y le ató por mil años; \footnote{\textbf{20:2} Apoc
  12,9} \bibverse{3} Y arrojólo al abismo, y le encerró, y selló sobre
él, porque no engañe más á las naciones, hasta que mil años sean
cumplidos: y después de esto es necesario que sea desatado un poco de
tiempo.

\bibverse{4} Y vi tronos, y se sentaron sobre ellos, y les fué dado
juicio; y vi las almas de los degollados por el testimonio de Jesús, y
por la palabra de Dios, y que no habían adorado la bestia, ni á su
imagen, y que no recibieron la señal en sus frentes, ni en sus manos; y
vivieron y reinaron con Cristo mil años. \footnote{\textbf{20:4} Apoc
  3,21; Mat 19,28; 1Cor 6,2} \bibverse{5} Mas los otros muertos no
tornaron á vivir hasta que sean cumplidos mil años. Esta es la primera
resurrección. \footnote{\textbf{20:5} 1Tes 4,16} \bibverse{6}
Bienaventurado y santo el que tiene parte en la primera resurrección: la
segunda muerte no tiene potestad en éstos; antes serán sacerdotes de
Dios y de Cristo, y reinarán con él mil años.

\hypertarget{gog-y-magog-apariciuxf3n-final-y-aniquilaciuxf3n-eterna-de-satanuxe1s-y-sus-huestes}{%
\subsection{Gog y Magog; aparición final y aniquilación eterna de
Satanás y sus
huestes}\label{gog-y-magog-apariciuxf3n-final-y-aniquilaciuxf3n-eterna-de-satanuxe1s-y-sus-huestes}}

\bibverse{7} Y cuando los mil años fueren cumplidos, Satanás será suelto
de su prisión, \bibverse{8} Y saldrá para engañar las naciones que están
sobre los cuatro ángulos de la tierra, á Gog y á Magog, á fin de
congregarlos para la batalla; el número de los cuales es como la arena
del mar. \footnote{\textbf{20:8} Ezeq 38,2} \bibverse{9} Y subieron
sobre la anchura de la tierra, y circundaron el campo de los santos, y
la ciudad amada: y de Dios descendió fuego del cielo, y los devoró.
\bibverse{10} Y el diablo que los engañaba, fué lanzado en el lago de
fuego y azufre, donde está la bestia y el falso profeta; y serán
atormentados día y noche para siempre jamás.

\hypertarget{la-segunda-resurrecciuxf3n-general-y-el-juicio-final}{%
\subsection{La segunda resurrección (general) y el juicio
final}\label{la-segunda-resurrecciuxf3n-general-y-el-juicio-final}}

\bibverse{11} Y vi un gran trono blanco y al que estaba sentado sobre
él, de delante del cual huyó la tierra y el cielo; y no fué hallado el
lugar de ellos. \footnote{\textbf{20:11} Mat 25,31-46; 2Pe 3,7; 2Pe
  3,10; 2Pe 3,12} \bibverse{12} Y vi los muertos, grandes y pequeños,
que estaban delante de Dios; y los libros fueron abiertos: y otro libro
fué abierto, el cual es de la vida: y fueron juzgados los muertos por
las cosas que estaban escritas en los libros, según sus obras.
\footnote{\textbf{20:12} Juan 5,28-29} \bibverse{13} Y el mar dió los
muertos que estaban en él; y la muerte y el infierno dieron los muertos
que estaban en ellos; y fué hecho juicio de cada uno según sus obras.
\bibverse{14} Y el infierno y la muerte fueron lanzados en el lago de
fuego. Esta es la muerte segunda. \footnote{\textbf{20:14} 1Cor 15,26;
  1Cor 15,55}

\bibverse{15} Y el que no fué hallado escrito en el libro de la vida,
fué lanzado en el lago de fuego.

\hypertarget{section-20}{%
\section{21}\label{section-20}}

\bibverse{1} Y vi un cielo nuevo, y una tierra nueva: porque el primer
cielo y la primera tierra se fueron, y el mar ya no es.

\hypertarget{la-nueva-jerusaluxe9n-como-morada-de-dios-con-el-pueblo-y-la-promesa-y-el-juicio-de-dios}{%
\subsection{La nueva Jerusalén como morada de Dios con el pueblo y la
promesa y el juicio de
Dios}\label{la-nueva-jerusaluxe9n-como-morada-de-dios-con-el-pueblo-y-la-promesa-y-el-juicio-de-dios}}

\bibverse{2} Y yo Juan vi la santa ciudad, Jerusalem nueva, que
descendía del cielo, de Dios, dispuesta como una esposa ataviada para su
marido. \footnote{\textbf{21:2} Heb 12,22; Gal 4,26; Apoc 19,7-8}
\bibverse{3} Y oí una gran voz del cielo que decía: He aquí el
tabernáculo de Dios con los hombres, y morará con ellos; y ellos serán
su pueblo, y el mismo Dios será su Dios con ellos. \footnote{\textbf{21:3}
  Ezeq 37,26-27} \bibverse{4} Y limpiará Dios toda lágrima de los ojos
de ellos; y la muerte no será más; y no habrá más llanto, ni clamor, ni
dolor: porque las primeras cosas son pasadas. \footnote{\textbf{21:4}
  Apoc 7,17; Is 25,8; Is 35,10}

\bibverse{5} Y el que estaba sentado en el trono dijo: He aquí, yo hago
nuevas todas las cosas. Y me dijo: Escribe; porque estas palabras son
fieles y verdaderas. \bibverse{6} Y díjome: Hecho es. Yo soy Alpha y
Omega, el principio y el fin. Al que tuviere sed, yo le daré de la
fuente del agua de vida gratuitamente. \bibverse{7} El que venciere,
poseerá todas las cosas; y yo seré su Dios, y él será mi hijo.
\bibverse{8} Mas á los temerosos é incrédulos, á los abominables y
homicidas, á los fornicarios y hechiceros, y á los idólatras, y á todos
los mentirosos, su parte será en el lago ardiendo con fuego y azufre,
que es la muerte segunda.

\hypertarget{la-descripciuxf3n-de-la-nueva-jerusaluxe9n}{%
\subsection{La descripción de la nueva
Jerusalén}\label{la-descripciuxf3n-de-la-nueva-jerusaluxe9n}}

\bibverse{9} Y vino á mí uno de los siete ángeles que tenían las siete
copas llenas de las siete postreras plagas, y habló conmigo, diciendo:
Ven acá, yo te mostraré la esposa, mujer del Cordero. \footnote{\textbf{21:9}
  Apoc 15,1; Apoc 15,6-7; Apoc 19,7} \bibverse{10} Y llevóme en Espíritu
á un grande y alto monte, y me mostró la grande ciudad santa de
Jerusalem, que descendía del cielo de Dios, \bibverse{11} Teniendo la
claridad de Dios: y su luz era semejante á una piedra preciosísima, como
piedra de jaspe, resplandeciente como cristal. \bibverse{12} Y tenía un
muro grande y alto con doce puertas; y en las puertas, doce ángeles, y
nombres escritos, que son los de las doce tribus de los hijos de Israel.
\bibverse{13} Al oriente tres puertas; al norte tres puertas; al
mediodía tres puertas; al poniente tres puertas. \bibverse{14} Y el muro
de la ciudad tenía doce fundamentos, y en ellos los doce nombres de los
doce apóstoles del Cordero.

\bibverse{15} Y el que hablaba conmigo, tenía una medida de una caña de
oro para medir la ciudad, y sus puertas, y su muro. \footnote{\textbf{21:15}
  Ezeq 40,3} \bibverse{16} Y la ciudad está situada y puesta en cuadro,
y su largura es tanta como su anchura: y él midió la ciudad con la caña,
doce mil estadios: la largura y la altura y la anchura de ella son
iguales. \bibverse{17} Y midió su muro, ciento cuarenta y cuatro codos,
de medida de hombre, la cual es del ángel. \bibverse{18} Y el material
de su muro era de jaspe: mas la ciudad era de oro puro, semejante al
vidrio limpio. \bibverse{19} Y los fundamentos del muro de la ciudad
estaban adornados de toda piedra preciosa. El primer fundamento era
jaspe; el segundo, zafiro; el tercero, calcedonia; el cuarto, esmeralda;
\bibverse{20} El quinto, sardónica; el sexto, sardio; el séptimo,
crisólito; el octavo, berilo; el nono, topacio; el décimo, crisopraso;
el undécimo, jacinto; el duodécimo, amatista. \bibverse{21} Y las doce
puertas eran doce perlas, en cada una, una; cada puerta era de una
perla. Y la plaza de la ciudad era de oro puro como vidrio trasparente.

\bibverse{22} Y no vi en ella templo; porque el Señor Dios Todopoderoso
es el templo de ella, y el Cordero. \bibverse{23} Y la ciudad no tenía
necesidad de sol, ni de luna, para que resplandezcan en ella: porque la
claridad de Dios la iluminó, y el Cordero era su lumbrera. \footnote{\textbf{21:23}
  Is 60,3; Is 60,5; Is 60,11; Is 60,19-20}

\bibverse{24} Y las naciones que hubieren sido salvas andarán en la
lumbre de ella: y los reyes de la tierra traerán su gloria y honor á
ella. \bibverse{25} Y sus puertas nunca serán cerradas de día, porque
allí no habrá noche. \bibverse{26} Y llevarán la gloria y la honra de
las naciones á ella. \bibverse{27} No entrará en ella ninguna cosa
sucia, ó que hace abominación y mentira; sino solamente los que están
escritos en el libro de la vida del Cordero.

\hypertarget{la-corriente-de-la-vida-los-uxe1rboles-de-la-vida-la-comuniuxf3n-plena-con-dios-y-el-reino-eterno-de-la-luz}{%
\subsection{La corriente de la vida, los árboles de la vida, la comunión
plena con Dios y el reino eterno de la
luz}\label{la-corriente-de-la-vida-los-uxe1rboles-de-la-vida-la-comuniuxf3n-plena-con-dios-y-el-reino-eterno-de-la-luz}}

\hypertarget{section-21}{%
\section{22}\label{section-21}}

\bibverse{1} Después me mostró un río limpio de agua de vida,
resplandeciente como cristal, que salía del trono de Dios y del Cordero.
\footnote{\textbf{22:1} Ezeq 47,1; Ezeq 47,12; Zac 14,8; Gén 2,9}
\bibverse{2} En el medio de la plaza de ella, y de la una y de la otra
parte del río, estaba el árbol de la vida, que lleva doce frutos, dando
cada mes su fruto: y las hojas del árbol eran para la sanidad de las
naciones. \bibverse{3} Y no habrá más maldición; sino que el trono de
Dios y del Cordero estará en ella, y sus siervos le servirán.
\bibverse{4} Y verán su cara; y su nombre estará en sus frentes.
\footnote{\textbf{22:4} Apoc 3,12} \bibverse{5} Y allí no habrá más
noche; y no tienen necesidad de lumbre de antorcha, ni de lumbre de sol:
porque el Señor Dios los alumbrará: y reinarán para siempre jamás.

\hypertarget{promesa-de-cristo-y-testimonio-de-juan-la-adoraciuxf3n-no-se-debe-a-la-criatura-sino-solo-a-dios}{%
\subsection{Promesa de Cristo y testimonio de Juan; La adoración no se
debe a la criatura, sino solo a
Dios}\label{promesa-de-cristo-y-testimonio-de-juan-la-adoraciuxf3n-no-se-debe-a-la-criatura-sino-solo-a-dios}}

\bibverse{6} Y me dijo: Estas palabras son fieles y verdaderas. Y el
Señor Dios de los santos profetas ha enviado su ángel, para mostrar á
sus siervos las cosas que es necesario que sean hechas presto.

\bibverse{7} Y he aquí, vengo presto. Bienaventurado el que guarda las
palabras de la profecía de este libro.

\bibverse{8} Yo Juan soy el que ha oído y visto estas cosas. Y después
que hube oído y visto, me postré para adorar delante de los pies del
ángel que me mostraba estas cosas. \bibverse{9} Y él me dijo: Mira que
no lo hagas: porque yo soy siervo contigo, y con tus hermanos los
profetas, y con los que guardan las palabras de este libro. Adora á
Dios.

\hypertarget{direcciuxf3n-y-amonestaciuxf3n-del-uxe1ngel-y-discurso-de-jesuxfas}{%
\subsection{Dirección y amonestación del ángel y discurso de
Jesús}\label{direcciuxf3n-y-amonestaciuxf3n-del-uxe1ngel-y-discurso-de-jesuxfas}}

\bibverse{10} Y me dijo: No selles las palabras de la profecía de este
libro; porque el tiempo está cerca. \footnote{\textbf{22:10} Apoc 1,3;
  Apoc 10,4} \bibverse{11} El que es injusto, sea injusto todavía: y el
que es sucio, ensúciese todavía: y el que es justo, sea todavía
justificado: y el santo sea santificado todavía. \footnote{\textbf{22:11}
  Dan 12,10}

\bibverse{12} Y he aquí, yo vengo presto, y mi galardón conmigo, para
recompensar á cada uno según fuere su obra. \footnote{\textbf{22:12} Is
  40,10} \bibverse{13} Yo soy Alpha y Omega, principio y fin, el primero
y el postrero. \footnote{\textbf{22:13} Apoc 1,11; Heb 13,8}
\bibverse{14} Bienaventurados los que guardan sus mandamientos, para que
su potencia sea en el árbol de la vida, y que entren por las puertas en
la ciudad. \footnote{\textbf{22:14} Apoc 7,14} \bibverse{15} Mas los
perros estarán fuera, y los hechiceros, y los disolutos, y los
homicidas, y los idólatras, y cualquiera que ama y hace mentira.
\footnote{\textbf{22:15} Apoc 21,8; Apoc 21,27; 1Cor 6,9-10}

\hypertarget{las-palabras-finales-de-jesuxfas-testimonio-del-espuxedritu-profuxe9tico-y-de-la-iglesia-la-ordenaciuxf3n-de-juan-para-su-libro-seguro-auxf1oranza-y-despedida}{%
\subsection{Las palabras finales de Jesús; Testimonio del espíritu
profético y de la iglesia; La ordenación de Juan para su libro; Seguro,
añoranza y
despedida}\label{las-palabras-finales-de-jesuxfas-testimonio-del-espuxedritu-profuxe9tico-y-de-la-iglesia-la-ordenaciuxf3n-de-juan-para-su-libro-seguro-auxf1oranza-y-despedida}}

\bibverse{16} Yo Jesús he enviado mi ángel para daros testimonio de
estas cosas en las iglesias. Yo soy la raíz y el linaje de David, la
estrella resplandeciente, y de la mañana. \footnote{\textbf{22:16} Is
  11,10; Luc 1,78}

\bibverse{17} Y el Espíritu y la Esposa dicen: Ven. Y el que oye, diga:
Ven. Y el que tiene sed, venga: y el que quiere, tome del agua de la
vida de balde. \footnote{\textbf{22:17} Juan 7,37; Is 55,1}

\bibverse{18} Porque yo protesto á cualquiera que oye las palabras de la
profecía de este libro: Si alguno añadiere á estas cosas, Dios pondrá
sobre él las plagas que están escritas en este libro. \bibverse{19} Y si
alguno quitare de las palabras del libro de esta profecía, Dios quitará
su parte del libro de la vida, y de la santa ciudad, y de las cosas que
están escritas en este libro. \bibverse{20} El que da testimonio de
estas cosas, dice: Ciertamente, vengo en breve. Amén, sea así. Ven,
Señor Jesús. \footnote{\textbf{22:20} 1Cor 16,22}

\bibverse{21} La gracia de nuestro Señor Jesucristo sea con todos
vosotros. Amén.
